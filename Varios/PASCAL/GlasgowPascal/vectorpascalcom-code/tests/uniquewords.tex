\documentclass[10pt, a4paper]{article}
\usepackage{graphicx, epsfig}
\reversemarginpar
\title{uniquewords.pas}
\begin{document}
\maketitle

\tableofcontents
\section{uniquewords}
\begin{tabbing}
***\=***\=***\=***\=***\=***\=***\=***\=***\=***\=***\=***\=***\=\kill
\\
\parbox{14cm}{\textsf{\textbf{program}  \textit{uniquewords} ;}}\\
\\
\\
\+\parbox{14cm}{\textsf{\textbf{const} }}\\
\parbox{14cm}{\textsf{\textit{wordmax} =20;}}\\
\<\parbox{14cm}{\textsf{\textbf{type} }}\\
\parbox{14cm}{\textsf{\textit{lexeme} =\textbf{string} [\textit{wordmax} ];}}\\
\parbox{14cm}{\textsf{\textit{lexset} = \textbf{set}  \textbf{of}  \textit{lexeme} ;}}\\
\\
\<\parbox{14cm}{\textsf{\textbf{var} }}\\
\parbox{14cm}{files: \textbf{ array } \textsf{[1..2] } \textbf{ of } \textsf{ \textit{text} ;}}\\
\parbox{14cm}{words: \textbf{ array } \textsf{[1..2] } \textbf{ of } \textsf{ \textit{lexset} ;}}\\
\parbox{14cm}{\textsf{Let \textit{i} $\in$ integer;}}\\
\parbox{14cm}{\textsf{Let \textit{commonwords} $\in$ lexset;}}\\
\parbox{14cm}{\textsf{Let \textit{aword} $\in$ lexeme;}}\\
\<\textsf{\textbf{function}  \textit{openfiles} :\textit{boolean} ;} (see Section \ref{sec:./uniquewordsopenfiles} )\\
\\
\<\textsf{\textbf{procedure}  \textit{loadset} \textit{(} \textbf{var}  \textit{f} :\textit{text} ;\textbf{var}  \textit{words} :\textit{lexset} );} (see Section \ref{sec:./uniquewordsloadset} )\\
\\
\\
\\
\\
\end{tabbing}
The main program reads in the files to sets, intersects them
and then lists the result.
For instance the command:
\begin{verbatim}
D:\WPC\documents\ilcg\book\tests>uniquewords norm.pas roman.pas
\end{verbatim}
produces the list of words
\begin{verbatim}
array
of
program
var
writeln
\end{verbatim}
The contents of the two files can be determined
by using the cat command
\begin{verbatim}
D:\WPC\documents\ilcg\book\tests>cat roman.pas norm.pas
program roman;
const rom:array[0..4] of string[1]=('C','L','X','V','I');
numb:array[0..4] of integer=(2,1,1,0,3);
var s:string;
s:= numb . rom;
writeln(s);
end.
\begin{tabbing}
***\=***\=***\=***\=***\=***\=***\=***\=***\=***\=***\=***\=***\=\kill
\+ \\
\\
\end{tabbing}
program norm;
type vec=array[0..3] of real;
function n(var v:vec):real;
n:= sqrt(\+ (v*v));
end;
var v:vec;    r:real;
v:=iota 0;
r:=n(v);
writeln(v,r);
end.
\end{verbatim}
On the other hand we can find all of the unique words in a single
file by intersecting it with itself thus:
\begin{verbatim}
D:\WPC\documents\ilcg\book\tests>uniquewords norm.pas norm.pas
array
function
iota
n
norm
of
program
r
real
sqrt
type
v
var
vec
writeln
\end{verbatim}
\begin{tabbing}
***\=***\=***\=***\=***\=***\=***\=***\=***\=***\=***\=***\=***\=\kill
\+ \\
\\
\\
\-\<\+\parbox{14cm}{\textsf{\textbf{begin} }}\\
\parbox{14cm}{\textsf {\textbf {if } \textsf{\textit{openfiles}}}}\\
\+\parbox{14cm}{\textsf{\textbf{then} }}\\
\<\parbox{14cm}{\textsf{\textbf{begin} }}\\
\parbox{14cm}{\textsf {\textbf {for } \textsf{\textit{i}$\leftarrow$ 1} \textbf{ to } \textsf{2} \textbf{ do } \textsf{\textit{loadset} (\textit{files}$_{\textit{i}}$, \textit{words}$_{\textit{i}}$)}; }}\\
\parbox{14cm}{\textsf{\textit{commonwords}$\leftarrow$ \textit{words}$_{1}$ $\times$ \textit{words}$_{2}$}; }\\
\+\parbox{14cm}{\textsf {\textbf {for } \textsf{\textit{aword}} \textbf{ in } \textsf{\textit{commonwords}} \textbf{ do } }}\\
\-\parbox{14cm}{\textsf{\textbf{writeln}(\textit{aword})}; }\\
\<\-\parbox{14cm}{\textsf{\textbf{end} }}\\
\parbox{14cm}{\textsf {\textbf {else } \textsf{\textbf{writeln}(\textit{\textrm{\textup { `Usage : uniquewords file1 file2' } }})}; }}\\
\<\-\parbox{14cm}{\textsf{\textbf{end} .}}\\
\end{tabbing}
\section{openfiles}\label{sec:./uniquewordsopenfiles}

\begin{tabbing}
***\=***\=***\=***\=***\=***\=***\=***\=***\=***\=***\=***\=***\=\kill
\parbox{14cm}{\textsf{\textbf{function}  \textit{openfiles} :\textit{boolean} ;}}\\
\end{tabbing}
This returns true if it has suceeded in opening both files.
Two possible error conditions can arise:
\begin{enumerate}
\item The number of filenames supplied to the program may be wrong.
This is tested using the integer valued function \textsf{{paramcount}\ }
which is provided in the System Unit. This returns the number of parameters
provided to the program on the command line.
\item The names provided may not  correspond to  valid files.
This is tested by attempting to \textsf{{reset}\ } the files for writing
and then testing the ioresult function. To use this one must disable the
automatic i/o checks provided on reseting a file which would otherwise
cause the program to abort with a run time error. This is done with
the compiler directive \{\$i-\}.
The previous presence of this directive allows the \textsf{ ioresult \ } function
to be used to test whether file opening failed.
\end{enumerate}
\begin{tabbing}
***\=***\=***\=***\=***\=***\=***\=***\=***\=***\=***\=***\=***\=\kill
\\
\\
\parbox{14cm}{\textsf{\textbf{label}  99;}}\\
\+\parbox{14cm}{\textsf{\textbf{var} }}\\
\parbox{14cm}{\textsf{Let \textit{i} $\in$ integer;}}\\
\-\<\+\parbox{14cm}{\textsf{\textbf{begin} }}\\
\parbox{14cm}{\textsf{\textit{openfiles}$\leftarrow$ \textit{false}}; }\\
\parbox{14cm}{\textsf {\textbf {if } \textsf{\textit{paramcount} $<$ 2}}}\\
\parbox{14cm}{\textsf {\textbf {then }  \textbf{ goto } \textsf{99}; }}\\
\+\parbox{14cm}{\textsf {\textbf {for } \textsf{\textit{i}$\leftarrow$ 1} \textbf{ to } \textsf{2} \textbf{ do } }}\\
\<\parbox{14cm}{\textsf{\textbf{begin} }}\\
\parbox{14cm}{\textsf{\textit{assign} (\textit{files}$_{\textit{i}}$, \textit{paramstr} (\textit{i}))}; }\\
\parbox{14cm}{\texttt{\small{\{\$i- checks off\}}}}\\
\parbox{14cm}{\textsf{\textit{reset} (\textit{files}$_{\textit{i}}$)}; }\\
\parbox{14cm}{\textsf {\textbf {if } \textsf{\textit{ioresult} $\neq$ 0}}}\\
\parbox{14cm}{\textsf {\textbf {then }  \textbf{ goto } \textsf{99}; }}\\
\parbox{14cm}{\texttt{\small{\{\$i+ checks on\}}}}\\
\<\-\parbox{14cm}{\textsf{\textbf{end} ;}}\\
\parbox{14cm}{\textsf{\textit{openfiles}$\leftarrow$ \textit{true}}; }\\
\parbox{14cm}{\textsf{99:}}\\
\<\-\parbox{14cm}{\textsf{\textbf{end} ;}}\\
\end{tabbing}
\section{loadset}\label{sec:./uniquewordsloadset}

\begin{tabbing}
***\=***\=***\=***\=***\=***\=***\=***\=***\=***\=***\=***\=***\=\kill
\parbox{14cm}{\textsf{\textbf{procedure}  \textit{loadset} \textit{(} \textbf{var}  \textit{f} :\textit{text} ;\textbf{var}  \textit{words} :\textit{lexset} );}}\\
\end{tabbing}
This procedure finds all the unique words in a file
and returns them in lexset.
This module is responsible for all of the parsing of
the input files.
It declares the set \textsf{letters} used in discriminating words
from other text.
\begin{tabbing}
***\=***\=***\=***\=***\=***\=***\=***\=***\=***\=***\=***\=***\=\kill
\\
\\
\+\parbox{14cm}{\textsf{\textbf{const} }}\\
\parbox{14cm}{\textsf{\textit{a} =\textrm{\textup { `a' } };}}\\
\parbox{14cm}{\textsf{\textit{z} =\textrm{\textup { `z' } };}}\\
\<\parbox{14cm}{\textsf{\textbf{var} }}\\
\parbox{14cm}{\textsf{Let \textit{letters} $\in$ set of char ;}}\\
\\
\<\parbox{14cm}{\textsf{\textbf{type} }}\\
\parbox{14cm}{\textsf{\textit{state}  = \textit{(} \textit{inword} , \textit{skipping} );}}\\
\<\parbox{14cm}{\textsf{\textbf{var} }}\\
\parbox{14cm}{\textsf{Let \textit{c} $\in$  char;}}\\
\parbox{14cm}{\textsf{Let \textit{s} $\in$  state;}}\\
\parbox{14cm}{\textsf{Let \textit{theword} $\in$ lexeme;}}\\
\<\textsf{\textbf{function}  \textit{getch} :\textit{char} ;} (see Section \ref{sec:./uniquewords/loadsetgetch} )\\
\<\textsf{\textbf{procedure}  \textit{getlex} \textit{(} \textbf{var}  \textit{l} :\textit{lexeme} );} (see Section \ref{sec:./uniquewords/loadsetgetlex} )\\
\\
\-\<\+\parbox{14cm}{\textsf{\textbf{begin} }}\\
\parbox{14cm}{\textsf{\textit{s}$\leftarrow$ \textit{skipping}}; }\\
\parbox{14cm}{\textsf{\textit{words}$\leftarrow$ \textit{}}; }\\
\parbox{14cm}{\textsf{\textit{letters}$\leftarrow$ \textit{}[\textrm{\textup { `a' } }..\textrm{\textup { `z' } }, \textrm{\textup { `A' } }..\textrm{\textup { `Z' } }]}; }\\
\+\parbox{14cm}{\textsf{\textbf{repeat} }}\\
\parbox{14cm}{\textsf{\textit{getlex} (\textit{theword})}; }\\
\parbox{14cm}{\textsf{\textit{words}$\leftarrow$ \textit{words} + \textit{}[theword]}; }\\
\-\<\parbox{14cm}{\textsf {\textbf {until } \textsf{\textit{theword} $=$ \textit{\textrm{\textup { `' } }}}; }}\\
\<\-\parbox{3.5cm}{\scriptsize{of loadset}}\'\parbox{14cm}{\textsf{\textbf{end}  ;}}\\
\end{tabbing}
\section{getch}\label{sec:./uniquewords/loadsetgetch}

\begin{tabbing}
***\=***\=***\=***\=***\=***\=***\=***\=***\=***\=***\=***\=***\=\kill
\parbox{14cm}{\textsf{\textbf{function}  \textit{getch} :\textit{char} ;}}\\
\end{tabbing}
Read in a character from the current file, return the null
character on end of file.
This function has to deal with the problems of
\begin{enumerate}
\item End of lines, which in Pascal are detected by the \textsf{{eoln}\ } function.
These are dealt with by returning the ASCII CR character 13.
\item End of file, detected by the \textsf{{eof}\ } function. This is
dealt with by returning the ASCII NUL character 0. The occurence of
NUL characters is dealt with at the next higher level of processing to
ensure that termination occurs.
\end{enumerate}
\begin{tabbing}
***\=***\=***\=***\=***\=***\=***\=***\=***\=***\=***\=***\=***\=\kill
\\
\\
\+\parbox{14cm}{\textsf{\textbf{var} }}\\
\parbox{14cm}{\textsf{Let \textit{local} $\in$ char;}}\\
\-\<\+\parbox{14cm}{\textsf{\textbf{begin} }}\\
\+\parbox{14cm}{\textsf {\textbf {if } \textsf{\textit{eoln} (\textit{f})} \textbf{ then } }}\\
\<\parbox{14cm}{\textsf{\textbf{begin} }}\\
\parbox{14cm}{\textsf{\textbf{readln} \textit{(} \textit{f} );}}\\
\parbox{14cm}{\textsf{\textit{getch}$\leftarrow$ \textbf{chr}(13)}; }\\
\<\-\parbox{14cm}{\textsf{\textbf{end} }}\\
\+\parbox{14cm}{\textsf{\textbf{else} }}\\
\<\parbox{14cm}{\textsf{\textbf{begin} }}\\
\+\parbox{14cm}{\textsf {\textbf {if } \textsf{\textit{eof} (\textit{f})} \textbf{ then } }}\\
\<\parbox{14cm}{\textsf{\textbf{begin} }}\\
\parbox{14cm}{\textsf{\textit{getch}$\leftarrow$ \textbf{chr}(0)}; }\\
\<\-\parbox{14cm}{\textsf{\textbf{end} }}\\
\+\parbox{14cm}{\textsf{\textbf{else} }}\\
\<\parbox{14cm}{\textsf{\textbf{begin} }}\\
\parbox{14cm}{\textsf{\textbf{read} \textit{(} \textit{f} ,\textit{local} );}}\\
\parbox{14cm}{\textsf{\textit{getch}$\leftarrow$ \textit{local}}; }\\
\<\-\parbox{14cm}{\textsf{\textbf{end} ;}}\\
\<\-\parbox{14cm}{\textsf{\textbf{end} ;}}\\
\<\-\parbox{3.5cm}{\scriptsize{of getch}}\'\parbox{14cm}{\textsf{\textbf{end}  ;}}\\
\end{tabbing}
\section{getlex}\label{sec:./uniquewords/loadsetgetlex}

\begin{tabbing}
***\=***\=***\=***\=***\=***\=***\=***\=***\=***\=***\=***\=***\=\kill
\parbox{14cm}{\textsf{\textbf{procedure}  \textit{getlex} \textit{(} \textbf{var}  \textit{l} :\textit{lexeme} );}}\\
\end{tabbing}
This procedure parses the input stream for the next word.
It then returns it in l. It operates as a simple finite
state machine that can be in one of two states:
\begin{enumerate}
\item \textsf{skipping} : the machine is in this state between words
whilst it moves over non letter characters.
\item \textsf{inword} : the machine is in this state whilst it parses
a word.
\end{enumerate}
The special case of the occurence of the null character
causes a branch to label 99 ensuring that a null string
is returned by the procedure. This is used at the
next higher level as a termination condition.
Labels, though deprecated in structured programming remain
a useful construct for escaping from loops.
\begin{tabbing}
***\=***\=***\=***\=***\=***\=***\=***\=***\=***\=***\=***\=***\=\kill
\\
\\
\end{tabbing}
Note that membership of the character in the set of letters
is used to switch between the two states of the parser.
This is an entirely orthodox use of sets in Pascal.
\begin{tabbing}
***\=***\=***\=***\=***\=***\=***\=***\=***\=***\=***\=***\=***\=\kill
\\
\\
\parbox{14cm}{\textsf{\textbf{label}  99;}}\\
\+\parbox{14cm}{\textsf{\textbf{begin} }}\\
\parbox{14cm}{\textsf{\textit{l}$\leftarrow$ \textit{\textrm{\textup { `' } }}}; }\\
\+\parbox{14cm}{\textsf {\textbf {while } \textsf{\textit{s} $=$ \textit{skipping}} \textbf{ do } }}\\
\<\parbox{14cm}{\textsf{\textbf{begin} }}\\
\parbox{14cm}{\textsf{\textit{c}$\leftarrow$ \textit{getch}}; }\\
\parbox{14cm}{\textsf {\textbf {if } \textsf{\textit{c}} \textbf{ in } \textsf{\textit{letters}} \textbf{ then } \textsf{\textit{s}$\leftarrow$ \textit{inword}}; }}\\
\parbox{14cm}{\textsf {\textbf {if } \textsf{\textit{c} $=$ \textbf{chr}(0)} \textbf{ then }  \textbf{ goto } \textsf{99}; }}\\
\<\-\parbox{14cm}{\textsf{\textbf{end} ;}}\\
\+\parbox{14cm}{\textsf {\textbf {while } \textsf{\textit{s} $=$ \textit{inword}} \textbf{ do } }}\\
\<\parbox{14cm}{\textsf{\textbf{begin} }}\\
\parbox{14cm}{\textsf {\textbf {if } \textsf{\textit{length} (\textit{l}) $=$ \textit{wordmax}}}}\\
\parbox{14cm}{\textsf {\textbf {then }  \textbf{ goto } \textsf{99}; }}\\
\parbox{14cm}{\textsf{\textit{l}$\leftarrow$ \textit{l} + \textit{c}}; }\\
\parbox{14cm}{\textsf{\textit{c}$\leftarrow$ \textit{getch}}; }\\
\parbox{14cm}{\textsf {\textbf {if } \textsf{\textit{c}} \textbf{ in } \textsf{\textit{letters}} \textbf{ then } \textsf{\textit{s}$\leftarrow$ \textit{inword}} \textbf{ else } \textsf{\textit{s}$\leftarrow$ \textit{skipping}}; }}\\
\parbox{14cm}{\textsf {\textbf {if } \textsf{\textit{c} $=$ \textbf{chr}(0)} \textbf{ then }  \textbf{ goto } \textsf{99}; }}\\
\<\-\parbox{14cm}{\textsf{\textbf{end} ;}}\\
\parbox{14cm}{\textsf{99:}}\\
\<\-\parbox{3.5cm}{\scriptsize{of getlex}}\'\parbox{14cm}{\textsf{\textbf{end}  ;}}\\
\end{tabbing}
\end{document}
