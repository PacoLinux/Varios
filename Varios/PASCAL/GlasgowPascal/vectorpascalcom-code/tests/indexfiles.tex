\documentclass[10pt, a4paper]{article}
\usepackage{graphicx, epsfig}
\reversemarginpar
\title{indexfiles.pas}
\begin{document}
\maketitle

\tableofcontents
\section{indexfiles}
\begin{tabbing}
***\=***\=***\=***\=***\=***\=***\=***\=***\=***\=***\=***\=***\=\kill
\parbox{14cm}{\textsf{\textbf{program}  \textit{indexfiles} ;}}\\
\+\parbox{14cm}{\textsf{\textbf{uses}  \textit{bloomfilter} ;}}\\
\<\parbox{14cm}{\textsf{\textbf{const} }}\\
\parbox{14cm}{\textsf{\textit{dirsep} =\textrm{\textup { `$ \backslash $' } };}}\\
\parbox{14cm}{\textsf{\textit{wordmax} =25;}}\\
\parbox{14cm}{\textsf{\textit{above} =\textrm{\textup { `..' } };}}\\
\parbox{14cm}{\textsf{\textit{this} =\textrm{\textup { `.' } };}}\\
\<\parbox{14cm}{\textsf{\textbf{type} }}\\
\parbox{14cm}{\textsf{\textit{lexeme} =\textbf{string} ;}}\\
\<\parbox{14cm}{\textsf{\textbf{var} }}\\
\parbox{14cm}{\textsf{Let \textit{index} $\in$ file of filefilter;}}\\
\\
\<\textsf{\textbf{procedure}  \textit{loadset} \textit{(} \textbf{var}  \textit{f} :\textit{text} ;\textbf{var}  \textit{words} :\textit{bloom} );} (see Section \ref{sec:indexfilesloadset} )\\
\<\textsf{\textbf{procedure}  \textit{processfile} \textit{(} \textit{fn} ,\textit{path} :\textbf{string} );} (see Section \ref{sec:indexfilesprocessfile} )\\
\<\textsf{\textbf{procedure}  \textit{intodir} \textit{(} \textit{s} :\textbf{string} ; \textit{prefix} :\textbf{string} );} (see Section \ref{sec:indexfilesintodir} )\\
\-\<\+\parbox{14cm}{\textsf{\textbf{begin} }}\\
\parbox{14cm}{\textsf{\textit{assign} (\textit{index}, \textit{\textrm{\textup { `wordindex.ind' } }})}; }\\
\parbox{14cm}{\textsf{\textit{rewrite} (\textit{index})}; }\\
\parbox{14cm}{\textsf{\textit{intodir} (\textit{\textrm{\textup { `.' } }}, \textit{\textrm{\textup { `.' } }})}; }\\
\parbox{14cm}{\textsf{\textit{close} (\textit{index})}; }\\
\<\-\parbox{14cm}{\textsf{\textbf{end} .}}\\
\end{tabbing}
\section{loadset}\label{sec:indexfilesloadset}

\begin{tabbing}
***\=***\=***\=***\=***\=***\=***\=***\=***\=***\=***\=***\=***\=\kill
\parbox{14cm}{\textsf{\textbf{procedure}  \textit{loadset} \textit{(} \textbf{var}  \textit{f} :\textit{text} ;\textbf{var}  \textit{words} :\textit{bloom} );}}\\
\end{tabbing}
This procedure finds all the unique words in a file
and returns them in lexset.
This module is responsible for all of the parsing of
the input files.
It declares the set \textsf{letters} used in discriminating words
from other text.
\begin{tabbing}
***\=***\=***\=***\=***\=***\=***\=***\=***\=***\=***\=***\=***\=\kill
\\
\\
\+\parbox{14cm}{\textsf{\textbf{const} }}\\
\parbox{14cm}{\textsf{\textit{a} =\textrm{\textup { `a' } };}}\\
\parbox{14cm}{\textsf{\textit{z} =\textrm{\textup { `z' } };}}\\
\<\parbox{14cm}{\textsf{\textbf{var} }}\\
\parbox{14cm}{\textsf{Let \textit{letters} $\in$ set of char ;}}\\
\\
\<\parbox{14cm}{\textsf{\textbf{type} }}\\
\parbox{14cm}{\textsf{\textit{state}  = \textit{(} \textit{inword} , \textit{skipping} );}}\\
\<\parbox{14cm}{\textsf{\textbf{var} }}\\
\parbox{14cm}{\textsf{Let \textit{c} $\in$  char;}}\\
\parbox{14cm}{\textsf{Let \textit{s} $\in$  state;}}\\
\parbox{14cm}{\textsf{Let \textit{theword} $\in$ lexeme;}}\\
\<\textsf{\textbf{function}  \textit{getch} :\textit{char} ;} (see Section \ref{sec:indexfiles/loadsetgetch} )\\
\<\textsf{\textbf{procedure}  \textit{getlex} \textit{(} \textbf{var}  \textit{l} :\textit{lexeme} );} (see Section \ref{sec:indexfiles/loadsetgetlex} )\\
\\
\-\<\+\parbox{14cm}{\textsf{\textbf{begin} }}\\
\parbox{14cm}{\textsf{\textit{s}$\leftarrow$ \textit{skipping}}; }\\
\\
\parbox{14cm}{\textsf{\textit{letters}$\leftarrow$ \textit{}[\textrm{\textup { `a' } }..\textrm{\textup { `z' } }, \textrm{\textup { `A' } }..\textrm{\textup { `Z' } }]}; }\\
\+\parbox{14cm}{\textsf{\textbf{repeat} }}\\
\parbox{14cm}{\textsf{\textit{getlex} (\textit{theword})}; }\\
\parbox{14cm}{\textsf{\textit{setfilter} (\textit{theword}, \textit{words})}; }\\
\-\<\parbox{14cm}{\textsf {\textbf {until } \textsf{\textit{theword} $=$ \textit{\textrm{\textup { `' } }}}; }}\\
\<\-\parbox{3.5cm}{\scriptsize{of loadset}}\'\parbox{14cm}{\textsf{\textbf{end}  ;}}\\
\end{tabbing}
\section{processfile}\label{sec:indexfilesprocessfile}

\begin{tabbing}
***\=***\=***\=***\=***\=***\=***\=***\=***\=***\=***\=***\=***\=\kill
\parbox{14cm}{\textsf{\textbf{procedure}  \textit{processfile} \textit{(} \textit{fn} ,\textit{path} :\textbf{string} );}}\\
\end{tabbing}
this builds an index for file fn and adds it to the index
\begin{tabbing}
***\=***\=***\=***\=***\=***\=***\=***\=***\=***\=***\=***\=***\=\kill
\\
\+\parbox{14cm}{\textsf{\textbf{var} }}\\
\parbox{14cm}{\textsf{Let \textit{ff} $\in$ filefilter;}}\\
\parbox{14cm}{\textsf{Let \textit{f} $\in$ text;}}\\
\-\<\+\parbox{14cm}{\textsf{\textbf{begin} }}\\
\parbox{14cm}{\textsf{\textbf{writeln}(\textit{path})}; }\\
\parbox{14cm}{\textsf{\textit{assign} (\textit{f}, \textit{fn})}; }\\
\parbox{14cm}{\texttt{\small{\{\$i-\}}}}\\
\parbox{14cm}{\textsf{\textit{reset} (\textit{f})}; }\\
\+\parbox{14cm}{\textsf {\textbf {if } \textsf{\textit{ioresult} $=$ 0} \textbf{ then } }}\\
\<\parbox{14cm}{\textsf{\textbf{begin} }}\\
\parbox{14cm}{\texttt{\small{\{\$i+\}}}}\\
\parbox{14cm}{\textsf{\textit{ff.wordset}$\leftarrow$ \textit{}}; }\\
\parbox{14cm}{\textsf{\textit{loadset} (\textit{f}, \textit{ff.wordset})}; }\\
\parbox{14cm}{\textsf{\textit{ff.filename}$\leftarrow$ \textit{path}}; }\\
\parbox{14cm}{\textsf{\textbf{write}(\textit{index}, \textit{ff})}; }\\
\<\-\parbox{14cm}{\textsf{\textbf{end} }}\\
\parbox{14cm}{\textsf {\textbf {else } \textsf{\textbf{writeln}(\textit{\textrm{\textup { `cant open ' } }}, \textit{fn}, \textit{\textrm{\textup { `:' } }}, \textit{path})}; }}\\
\parbox{14cm}{\textsf{\textit{close} (\textit{f})}; }\\
\\
\<\-\parbox{14cm}{\textsf{\textbf{end} ;}}\\
\end{tabbing}
\section{intodir}\label{sec:indexfilesintodir}

\begin{tabbing}
***\=***\=***\=***\=***\=***\=***\=***\=***\=***\=***\=***\=***\=\kill
\parbox{14cm}{\textsf{\textbf{procedure}  \textit{intodir} \textit{(} \textit{s} :\textbf{string} ; \textit{prefix} :\textbf{string} );}}\\
\+\parbox{14cm}{\textsf{\textbf{var} }}\\
\parbox{14cm}{buf: \textbf{ array } \textsf{[0..100] } \textbf{ of } \textsf{ \textit{ascii} ;}}\\
\parbox{14cm}{\textsf{Let \textit{n} $\in$ pchar;}}\\
\parbox{14cm}{\textsf{Let \textit{un}, \textit{path} $\in$ string;}}\\
\parbox{14cm}{\textsf{Let \textit{thedir} $\in$ pdir;}}\\
\parbox{14cm}{\textsf{Let \textit{theentry} $\in$ pdirentry;}}\\
\-\<\+\parbox{14cm}{\textsf{\textbf{begin} }}\\
\parbox{14cm}{\textsf{\textit{unicodestring2ascii} (\textit{s}, \textit{buf}$_{0}$)}; }\\
\parbox{14cm}{\textsf{\textit{thedir} :=\textit{opendir} \textit{(} @\textit{buf} );}}\\
\+\<\parbox{14cm}{\textsf {\textbf {if } \textsf{\textit{thedir} $\neq$ \textit{nil}} \textbf{ then } \textsf{\textit{begin}} \textbf{ begin } }}\\
\parbox{14cm}{\textsf{\textit{chdir} \textit{(} @\textit{buf} );}}\\
\parbox{14cm}{\textsf{\textit{theentry}$\leftarrow$ \textit{readdir} (\textit{thedir})}; }\\
\+\parbox{14cm}{\textsf {\textbf {while } \textsf{(\textit{theentry} $\neq$ \textit{nil})} \textbf{ do } }}\\
\<\parbox{14cm}{\textsf{\textbf{begin} }}\\
\parbox{14cm}{\textsf{\textit{n}$\leftarrow$ \textit{entryname} (\textit{theentry})}; }\\
\parbox{14cm}{\textsf{\textit{un}$\leftarrow$ \textit{strpas} (\textit{n})}; }\\
\parbox{14cm}{\textsf{\textit{path}$\leftarrow$ \textit{prefix} + \textit{dirsep} + \textit{un}}; }\\
\+\<\parbox{14cm}{\textsf {\textbf {if } \textsf{\textit{isdir} (\textit{n})} \textbf{ then } \textsf{\textit{begin}} \textbf{ begin } }}\\
\+\parbox{14cm}{\textsf {\textbf {if } \textsf{\textit{un} $\neq$ \textit{above}} \textbf{ then } }}\\
\+\parbox{14cm}{\textsf {\textbf {if } \textsf{\textit{un} $\neq$ \textit{this}} \textbf{ then } }}\\
\-\-\parbox{14cm}{\textsf{\textit{intodir} (\textit{un}, \textit{path})}; }\\
\<\-\parbox{14cm}{\textsf{\textbf{end}  \textbf{else}  \textit{processfile}  \textit{(} \textit{un} ,\textit{path} );}}\\
\parbox{14cm}{\textsf{\textit{theentry}$\leftarrow$ \textit{readdir} (\textit{thedir})}; }\\
\<\-\parbox{14cm}{\textsf{\textbf{end} ;}}\\
\parbox{14cm}{\textsf{\textit{unicodestring2ascii} (\textit{above}, \textit{buf}$_{0}$)}; }\\
\parbox{14cm}{\textsf {\textbf {if } \textsf{\textit{s} $\neq$ \textit{\textrm{\textup { `.' } }}} \textbf{ then } \textsf{	\textit{chdir} \textit{(} @\textit{buf} );}}}\\
\<\-\parbox{14cm}{\textsf{\textbf{end} ;}}\\
\<\-\parbox{14cm}{\textsf{\textbf{end} ;}}\\
\end{tabbing}
\section{getch}\label{sec:indexfiles/loadsetgetch}

\begin{tabbing}
***\=***\=***\=***\=***\=***\=***\=***\=***\=***\=***\=***\=***\=\kill
\parbox{14cm}{\textsf{\textbf{function}  \textit{getch} :\textit{char} ;}}\\
\end{tabbing}
Read in a character from the current file, return the null
character on end of file.
This function has to deal with the problems of
\begin{enumerate}
\item End of lines, which in Pascal are detected by the \textsf{{eoln}\ } function.
These are dealt with by returning the ASCII CR character 13.
\item End of file, detected by the \textsf{{eof}\ } function. This is
dealt with by returning the ASCII NUL character 0. The occurence of
NUL characters is dealt with at the next higher level of processing to
ensure that termination occurs.
\end{enumerate}
\begin{tabbing}
***\=***\=***\=***\=***\=***\=***\=***\=***\=***\=***\=***\=***\=\kill
\\
\\
\+\parbox{14cm}{\textsf{\textbf{var} }}\\
\parbox{14cm}{\textsf{Let \textit{local} $\in$ char;}}\\
\-\<\+\parbox{14cm}{\textsf{\textbf{begin} }}\\
\+\parbox{14cm}{\textsf {\textbf {if } \textsf{\textit{eoln} (\textit{f})} \textbf{ then } }}\\
\<\parbox{14cm}{\textsf{\textbf{begin} }}\\
\parbox{14cm}{\textsf{\textbf{readln} \textit{(} \textit{f} );}}\\
\parbox{14cm}{\textsf{\textit{getch}$\leftarrow$ \textbf{chr}(13)}; }\\
\<\-\parbox{14cm}{\textsf{\textbf{end} }}\\
\+\parbox{14cm}{\textsf{\textbf{else} }}\\
\<\parbox{14cm}{\textsf{\textbf{begin} }}\\
\+\parbox{14cm}{\textsf {\textbf {if } \textsf{\textit{eof} (\textit{f})} \textbf{ then } }}\\
\<\parbox{14cm}{\textsf{\textbf{begin} }}\\
\parbox{14cm}{\textsf{\textit{getch}$\leftarrow$ \textbf{chr}(0)}; }\\
\<\-\parbox{14cm}{\textsf{\textbf{end} }}\\
\+\parbox{14cm}{\textsf{\textbf{else} }}\\
\<\parbox{14cm}{\textsf{\textbf{begin} }}\\
\parbox{14cm}{\textsf{\textbf{read} \textit{(} \textit{f} ,\textit{local} );}}\\
\parbox{14cm}{\textsf{\textit{getch}$\leftarrow$ \textit{local}}; }\\
\<\-\parbox{14cm}{\textsf{\textbf{end} ;}}\\
\<\-\parbox{14cm}{\textsf{\textbf{end} ;}}\\
\<\-\parbox{3.5cm}{\scriptsize{of getch}}\'\parbox{14cm}{\textsf{\textbf{end}  ;}}\\
\end{tabbing}
\section{getlex}\label{sec:indexfiles/loadsetgetlex}

\begin{tabbing}
***\=***\=***\=***\=***\=***\=***\=***\=***\=***\=***\=***\=***\=\kill
\parbox{14cm}{\textsf{\textbf{procedure}  \textit{getlex} \textit{(} \textbf{var}  \textit{l} :\textit{lexeme} );}}\\
\end{tabbing}
This procedure parses the input stream for the next word.
It then returns it in l. It operates as a simple finite
state machine that can be in one of two states:
\begin{enumerate}
\item \textsf{skipping} : the machine is in this state between words
whilst it moves over non letter characters.
\item \textsf{inword} : the machine is in this state whilst it parses
a word.
\end{enumerate}
The special case of the occurence of the null character
causes a branch to label 99 ensuring that a null string
is returned by the procedure. This is used at the
next higher level as a termination condition.
Labels, though deprecated in structured programming remain
a useful construct for escaping from loops.
\begin{tabbing}
***\=***\=***\=***\=***\=***\=***\=***\=***\=***\=***\=***\=***\=\kill
\\
\\
\end{tabbing}
Note that membership of the character in the set of letters
is used to switch between the two states of the parser.
This is an entirely orthodox use of sets in Pascal.
\begin{tabbing}
***\=***\=***\=***\=***\=***\=***\=***\=***\=***\=***\=***\=***\=\kill
\\
\\
\parbox{14cm}{\textsf{\textbf{label}  99;}}\\
\+\parbox{14cm}{\textsf{\textbf{begin} }}\\
\parbox{14cm}{\textsf{\textit{l}$\leftarrow$ \textit{\textrm{\textup { `' } }}}; }\\
\+\parbox{14cm}{\textsf {\textbf {while } \textsf{\textit{s} $=$ \textit{skipping}} \textbf{ do } }}\\
\<\parbox{14cm}{\textsf{\textbf{begin} }}\\
\parbox{14cm}{\textsf{\textit{c}$\leftarrow$ \textit{getch}}; }\\
\parbox{14cm}{\textsf {\textbf {if } \textsf{\textit{c}} \textbf{ in } \textsf{\textit{letters}} \textbf{ then } \textsf{\textit{s}$\leftarrow$ \textit{inword}}; }}\\
\parbox{14cm}{\textsf {\textbf {if } \textsf{\textit{c} $=$ \textbf{chr}(0)} \textbf{ then }  \textbf{ goto } \textsf{99}; }}\\
\<\-\parbox{14cm}{\textsf{\textbf{end} ;}}\\
\+\parbox{14cm}{\textsf {\textbf {while } \textsf{\textit{s} $=$ \textit{inword}} \textbf{ do } }}\\
\<\parbox{14cm}{\textsf{\textbf{begin} }}\\
\parbox{14cm}{\textsf {\textbf {if } \textsf{\textit{length} (\textit{l}) $=$ \textit{wordmax}}}}\\
\parbox{14cm}{\textsf {\textbf {then }  \textbf{ goto } \textsf{99}; }}\\
\parbox{14cm}{\textsf{\textit{l}$\leftarrow$ \textit{l} + \textit{c}}; }\\
\parbox{14cm}{\textsf{\textit{c}$\leftarrow$ \textit{getch}}; }\\
\\
\parbox{14cm}{\textsf {\textbf {if } \textsf{\textit{c}} \textbf{ in } \textsf{\textit{letters}} \textbf{ then } \textsf{\textit{s}$\leftarrow$ \textit{inword}} \textbf{ else } \textsf{\textit{s}$\leftarrow$ \textit{skipping}}; }}\\
\parbox{14cm}{\textsf {\textbf {if } \textsf{\textit{c} $=$ \textbf{chr}(0)} \textbf{ then }  \textbf{ goto } \textsf{99}; }}\\
\<\-\parbox{14cm}{\textsf{\textbf{end} ;}}\\
\parbox{14cm}{\textsf{99:}}\\
\<\-\parbox{3.5cm}{\scriptsize{of getlex}}\'\parbox{14cm}{\textsf{\textbf{end}  ;}}\\
\end{tabbing}
\section{bloomfilter}
\input{bloomfilter}

\end{document}
