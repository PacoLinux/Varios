%% LyX 1.1 created this file.  For more info, see http://www.lyx.org/.
%% Do not edit unless you really know what you are doing.
\documentclass{article}
\usepackage[T1]{fontenc}
\usepackage[latin1]{inputenc}

\makeatletter


%%%%%%%%%%%%%%%%%%%%%%%%%%%%%% LyX specific LaTeX commands.
\providecommand{\LyX}{L\kern-.1667em\lower.25em\hbox{Y}\kern-.125emX\@}

%%%%%%%%%%%%%%%%%%%%%%%%%%%%%% User specified LaTeX commands.
\usepackage{times}
\usepackage{a4}

\makeatother

\begin{document}


\title{Vector Pascal}


\author{Paul Cockshott, University of Glasgow}

\maketitle
\begin{abstract}
Vector Pascal is a language designed to support elegant and efficient expression
of algorithms using the SIMD model of computation. It imports into Pascal features
derived from the functional languages APL and J, in particular the extension
of all operators to work on vectors of data. The type system is extended to
handle dimensional analysis. Code generation is via the ILCG system that allows
retargeting to multiple different SIMD instructionsets based on formal descrition
of the instructionset semantics.
\end{abstract}

\section*{Introduction}

The introduction of SIMD instruction sets\cite{IA32 }\cite{AMD}\cite{Peleg97}\cite{Intel00}
to Personal Computers potentially provides substantial performance increases,
but the ability of most programmers to harness this performance is held back
by two factors. The first is the limited availability of compilers that make
effective use of these instructionsets in a machine independent manner. This
remains the case despite the research efforts to develop compilers for multi-media
instructionsets\cite{Cheong97}\cite{Leupers99}\cite{Krall00}\cite{Sreraman00}.
The second is the fact that most popular programming languages were designed
on the word at a time model of the classic von Neuman computer. Whilst processor
architectures are moving towards greater levels of parallelism, the most widely
used programming languages like C, Java and Delphi are structured around a model
of computation in which operations take place on a single value at a time. This
was appropriate when processors worked this way, but has become an impediment
to programmers seeking to make use of the performance offered by multi-media
instructionsets. These problems mean that it is, in practice, significantly
harder to write a program that will use SIMD features than it is to write a
conventional program. 

Vector Pascal aims to provide an efficient and concise notation for programmers
using Multi-Media enhanced CPUs. In doing so it borrows concepts for expressing
data parallelism that have a long history, dating back to Iverson's work on
APL in the early '60s\cite{Iverson62}. 

The\( T \)as having type \( T[] \). Then if we have a binary operator \( \omega :(T\otimes T)\rightarrow T \),
in languages derived from APL we automatically have an operator \( \omega :(T[]\otimes T[])\rightarrow T[] \)
\(  \). Thus if \( x,y \) are arrays of integers \( k=x+y \) is the array
of integers where \( k_{i}=x_{i}+y_{i} \).

The basic concept is simple, there are complications to do with the semantics
of operations between arrays of different lengths and different dimensions,
but Iverson provides a consistent treatment of these. The most recent languages
to be built round this model are J, an interpretive language\cite{Jmanual}\cite{Burke}\cite{Jintro},
and F\cite{Metcalf96} a modernised Fortran. In principle though any language
with array types can be extended in a similar way. Iverson's approach to data
parallelism is machine independent. It can be implemented using scalar instructions
or using the SIMD model. The only difference is speed. 

Vector Pascal incorporates Iverson's approach to data parallelism. Its aim is
to provide a notation that allows the natural and elegant expression of data
parallel algorithms within a base language that is already familiar to a considerable
body of programmers and combine this with modern compilation techniques. 

By an elegant algorithm I mean one which is expressed as concisely as possible.
Elegance is a goal that one approaches asymtotically, approaching but never
attaining\cite{Chaiten}. APL and J allow the construction of very elegant programs,
but at a cost. An inevitable consequence of elegance is the loss of redundancy.
APL programs are as concise, or even more concise than conventional mathematical
notation\cite{Iversion80} and use a special characterset. This makes them hard
for the uninitiated to understand. J attempts to remedy this by restricting
itself to the ASCII characterset, but still looks dauntingly unfamiliar to programmers
brought up on more conventional languages. Both APL and J are interpretive which
makes them ill suited to many of the applications for which SIMD speed is required.
The aim of Vector Pascal is to provide the conceptual gains of Iverson's notation
within a framework familiar to imperative programmers.

Pascal\cite{Wirth}was chosen as a base language over the alternatives of C
and Java. C was rejected because notations like \texttt{x+y} for \texttt{x}
and \texttt{y} declared as \texttt{int x{[}4{]},y{[}4{]}}, already have the
meaning of adding the addresses of the arrays together. Java was rejected because
of the difficulty of efficiently transmitting data parallel operations via its
intermediate code to a just in time code generator. 


\section*{Type System}

The type system of Pascal is extended by the provision of dynamic arrays and
by the introduction of dimensioned types. 


\subsection*{Dimensioned Types}

Dimensional analysis is familiar to scientists and engineers and provides a
routine check on the sanity of mathematical expressions. Dimensions can not
be expressed in the otherwise rigourous type system of standard Pascal, but
they are a useful protection against the sort of programming confusion between
imperial and metric units that caused the demise of a recent Mars probe. They
provide a means by which floating point types can be specialised to represent
dimensioned numbers as is required in physics calculations. For example:

\textbf{kms =(mass,distance,time);}

\textbf{meter=real of distance;}

\textbf{kilo=real of mass;}

\textbf{second=real of time;}

\textbf{newton=real of mass {*} distance {*} time POW -2;}

\textbf{meterpersecond = real of distance {*}time POW -1;}

The grammar is given by:

\vspace{0.3cm}
{\centering \begin{tabular}{|c|c|}
\hline 
<dimensioned type>&
<real type> <dimension >{[}'{*}' <dimension>{]}{*}\\
\hline 
\end{tabular}\par}
\vspace{0.3cm}

\vspace{0.3cm}
{\centering \begin{tabular}{|c|c|}
\hline 
<real type>&
'real'\\
&
'double'\\
\hline 
\end{tabular}\par}
\vspace{0.3cm}

\vspace{0.3cm}
{\centering \begin{tabular}{|c|c|}
\hline 
<dimension>&
<identifier> {[}'POW' {[}<sign>{]} <unsigned integer>{]}\\
\hline 
\end{tabular}\par}
\vspace{0.3cm}

The identifier must be a member of a scalar type, and that scalar type is then
refered to as the basis space of the dimensioned type. The identifiers of the
basis space are refered to as the dimensions of the dimensioned type. Associated
with each dimension of a dimensioned type there is an integer number refered
to as the power of that dimension. This is either introduced explicitly at type
declaration time, or determined implicitly for the dimensional type of expressions. 

A value of a dimensioned type is a dimensioned value. Let \( \log _{d}t \)
of a dimensioned type \( t \) be the power to which the dimension \( d \)
of type \( t \) is raised. Thus for \( t= \)newton in the example above, and
\( d= \)time, \( \log _{d}t=-2 \)

If \( x \) and \( y \) are values of dimensioned types \( t_{x} \)and \( t_{y} \)respectively,
then the following operators are only permissible if \( t_{x}=t_{y} \): +,
- ,<, >, =, <=, >=. For + and -, the dimensional type of the result is the same
as that of the arguments. The operations. The operations {*}, / are permited
if the types \( t_{x} \)and \( t_{y} \) share the same basis space, or if
the basis space of one of the types is a subrange of the basis space of the
other.  

The operation \textbf{POW} is permited between dimensioned types and integers.


\paragraph*{Dimension deduction rules}

\begin{enumerate}
\item If \( x=y*z \) for \( x:t_{1},y:t_{2},z:t_{3} \) with basis space \( B \)
then \( \forall _{d\in B}\log _{d}t_{1}=\log _{d}t_{2}+\log _{d}t_{3} \). 
\item If \( x=y/z \) for \( x:t_{1},y:t_{2},z:t_{3} \) with basis space \( B \)
then \( \forall _{d\in B}\log _{d}t_{1}=\log _{d}t_{2}-\log _{d}t_{3} \). 
\item If \( x=y \) \textbf{POW} \( z \) for \( x:t_{1},y:t_{2},z:integer \) with
basis space for \( t_{2} \), \( B \) then \( \forall _{d\in B}\log _{d}t_{1}=\log _{d}t_{2}\times z \).
\end{enumerate}

\subsection*{Dynamic Arrays}

A dynamic array is an array whose bounds are determined at run time. Operations
on dynamic arrays are essential in general purpose image processing software.

Pascal 90\cite{ISO90} introduced the notion of schematic or parameterised types
as a means of creating dynamic arrays. Thus where \textbf{r} is some integral
type one can write 

\textbf{type z(a,b:r)=array{[}a..b{]} of t;}

If \textbf{p:\textasciicircum{}z}, then 

\textbf{new(p,n,m)}

where \textbf{n,m:r} initialises \textbf{p} to point to an array of bounds \textbf{n..m}.
The bounds of the array can then be accessed as \textbf{p\textasciicircum{}.a,
p\textasciicircum{}.b}. Vector Pascal incorporates this notation from Pascal
90 for dynamic arrays. 


\section*{Expressions}

The expression syntax of Vector Pascal incorportates extensions to refer to
ranges of arrays and to operate on arrays as a whole.


\subsection*{Indexed Ranges}

A range of components of an array variable are denoted by the variable followed
by a range expression in brackets.

\vspace{0.3cm}
{\centering \begin{tabular}{|c|c|}
\hline 
<indexed range>&
<variable> '{[}' <range expression>{[}',' <range expression>{]}{*} '{]}'\\
\hline 
\end{tabular}\par}
\vspace{0.3cm}

\vspace{0.3cm}
{\centering \begin{tabular}{|c|c|}
\hline 
<range expression>&
<expression> '..' <expression>\\
\hline 
\end{tabular}\par}
\vspace{0.3cm}

The expressions within the range expression must conform to the index type of
the array variable. The type of a range expression \textbf{a{[}i..j{]}} where
\textbf{a: array{[}p..q{]} of t} is \textbf{array{[}0..j-i{]} of t.}

Examples

\textbf{dataset{[}i..i+2{]}:=blank;}

\textbf{twoDdata{[}2..3,5..6{]}:=twoDdata{[}4..5,11..12{]}{*}0.5;}

Subranges may be passed in as actual parameters to procedures whose corresponding
formal parameters are declared as variables of a schematic type. Hence given
the following declarations:

\textbf{type image(miny,maxy,minx,maxx:integer)=array{[}miny..maxy,minx..maxx{]}
of byte;}

\textbf{procedure invert(var im:image);begin im:=255-im; end;}

\textbf{var screen:array{[}0..319,0..199{]} of byte;}

then the following statement would be valid:

\textbf{invert(screen{[}40..60,20..30{]});}


\subsection*{Unary expressions}

A unary expression is formed by applying a unary operator to another unary or
primary expression. The unary operators supported are \textbf{+, -, {*}, /,
div, not, round, sqrt, sin, cos, tan, abs, ln, ord, chr, succ, pred} and \textbf{@}.

Thus the following are valid unary expressions\textbf{: -1, +b, not true, sqrt
abs x, sin theta.} In standard Pascal some of these operators are treated as
functions,. Syntactically this means that their arguments must be enclosed in
brackets, as in \textbf{sin(theta)}. This usage remains syntactically correct
in Vector Pascal. 

The dyadic operators \textbf{+, -, {*}, /, div} are all extended to unary context
by the insertion of an implicit value under the operation. Thus just as \textbf{-a
= 0-a} so too \textbf{/2 = 1/2}. For sets the notation \textbf{-s} means the
complement of the set \textbf{s}. The implicit value inserted are given below.

\vspace{0.3cm}
{\centering \begin{tabular}{|c|c|c|}
\hline 
type&
operators&
implicit value\\
\hline 
\hline 
number&
+,-&
0\\
\hline 
set&
+&
empty set\\
\hline 
set&
-,{*}&
fullset\\
\hline 
number&
{*},/ ,div&
1\\
\hline 
\end{tabular}\par}
\vspace{0.3cm}

A unary operator can be applied to an array argument and returns an array result.
Similarly any user declared function over a scalar type can be applied to an
array type and return an array. If \textbf{f} is a function or unary operator
mapping from type \textbf{r} to type \textbf{t} then if \textbf{x} is an array
of \textbf{r} then \textbf{a:=f(x)} assigns an array of \textbf{t} such that
\textbf{a{[}i{]}=f(x{[}i{]})}


\subsubsection*{Dyadic Operations}

Dyadic operators supported are \textbf{+, +:, -:, -, {*}, /, div, mod , {*}{*},
pow, <, >, >=, <=, =, <>, shr, shl, and, or, in}. All of these are consistently
extended to operate over arrays. The operators {*}{*}, pow denote exponentiation
and raising to an integer power. The operators +: and -: exist to support saturated
arithmetic as supported by the MMX instructionset.


\subsubsection*{Addition operations}

{\centering \begin{tabular}{ccccc}
\hline 
Operation&
Left&
Right&
Result&
Effect of \emph{a} \textbf{op} \emph{b}\\
\hline 
\hline 
+&
integer&
integer&
integer&
sum of \emph{a} and \emph{b}\\
&
real&
real&
real&
sum of \emph{a} and \emph{b}\\
-&
integer&
integer&
integer&
result of subtracting \emph{b} from \emph{a}\\
&
real&
real&
real&
result of subtracting \emph{b} from \emph{a}\\
+:&
0..255&
0..255&
0..255&
saturated addition cliped to 0..255 \\
&
-128..127&
-128..127&
-128..127&
saturated addition clipped to -128..127\\
-:&
0..255&
0..255&
0..255&
saturated subtraction clipped to 0..255\\
\hline 
&
-128..127&
-128..127&
-128..127&
saturated subtraction clipped to -128..127\\
\hline 
\end{tabular}\par}


\subsection*{Assignment}

Standard Pascal allows assignement of whole arrays. Vector Pascal extends this
to allow consistent use of mixed rank expressions on the right hand side of
an assignment. Given

\textbf{r0:real; r1:array{[}0..7{]} of real; r2:array{[}0..7,0..7{]} of real}

then we can write

\begin{enumerate}
\item r\textbf{1:= r2{[}3{]}; \{ supported in standard Pascal \}}
\item \textbf{r1:= /2; \{ assign 0.5 to each element of r1 \}}
\item \textbf{r2:= r1{*}3; \{ assign 1.5 to every element of r2\}}
\item \textbf{r1:= \textbackslash{}+ r2; \{ r1gets the totals along the rows of r2\}}
\end{enumerate}
The variable on the left hand side of an assignment defines an array context
within which expressions on the right hand side are evaluated. Each array context
has a rank given by the number of dimensions of the array on the left hand side.
A scalar variable has rank 0. Variables occuring in expressions with an array
context of rank \emph{r} must have \emph{r} or fewer dimensions. The \emph{n}
bounds of any \emph{n} dimensional array variable, with \( n\leq r \) occuring
within an expression evaluated in an array context of rank \emph{r} must match
with the rightmost \emph{n} bounds of the array on the left hand side of the
assignment statement. 

Where a variable is of lower rank than its array context, the variable is replicated
to fill the array context. This is shown in examples 2 and 3 above. Because
the rank of any assignment is constrained by the variable on the left hand side,
no temporary arrays, other than machine registers, need be allocated to store
the intermediate array results of expressions.


\subsubsection*{Operator Reduction}

Any dyadic operator can be converted to a monadic reduction operator by the
functional \textbackslash{}. Thus if a is an array, \textbackslash{}+a denotes
the sum over the array. More generally \( \setminus \Phi x \) for some dyadic
operator \( \Phi  \) means \( x_{0}\Phi (x_{1}\Phi ..(x_{n}\Phi \iota )) \)
where \( \iota  \) is the implicit value given the operator and the type. Thus
we can write \textbackslash{}+ for \( \sum  \), \textbackslash{}{*} for \( \prod  \)
etc. The dot product of two vectors can thus be written as

\textbf{x:=\textbackslash{}+(y{*}z);}

instead of

\textbf{x:=0;}

\textbf{for i:=0 to n do x:= x+ y{[}i{]}{*}z{[}i{]};}

A reduction operation takes an argument of rank \emph{r} and returns an argument
of rank \emph{r-1} except in the case where its argument is or rank 0, in which
case it acts as the identity operation. Reduction is always performed along
the last array dimension of its argument. 


\section*{Implementation}

The Vector Pascal compiler is implemented in Java. It uses the ILCG\cite{Cockshott00}
portable code generation system. A Vector Pascal program is translated into
an abstract semantic tree implemented as a Java datastructure. The tree is passed
to a machine generated Java class corresponding to the code generator for the
target machine. Code generator classes currently exist for the Intel 486, Pentium
with MMX, and P3 and also the AMD K6. Output is assembler code which is assembled
using the NASM assembler and linked using the gcc loader. Vector Pascal currently
runs under Windows98 , Windows2000 and Linux. Separate compilation using Turbo
Pascal style units is supported.

The code generators follow the pattern matching approach described in\cite{Aho}\cite{Cattel80}and
\cite{graham80}, and are automatically generated from machine specifications
written in ILCG (Intermediate Language for Code Generators). ILCG is a strongly
typed language which supports vector data types and operators over vectors.
It is well suited to describing MMX type instructionsets. The code generator
classes export from their interfaces details about the degree of parallelism
supported for each data-type. This is used by the front end compiler to iterate
over arrays longer than those supported by the underlying machine. Where supported
parallelism is unitary this defaults to iteration over the whole array.

Selection of target machines is by a compile time switch which causes the appropriate
code generator class to be dynamically loaded. 


\section*{Conclusions}

Vector Pascal provides a new approach to providing a programming environment
for multi-media instructionsets. It borrows notational conventions that have
a long history of sucessfull use in interpretive programming languages, combining
these with modern compiler techniques to target SIMD instructionsets. It provides
a uniform source language that can target multiple different processors without
the programmer having to think about the target machine. Use of Java as the
implementation language aids portability of the compiler accross operating systems.
Work is underway to compare the performance and elegance of implementations
of a stereo-matcher algorithm implemented in Vector Pascal with the same algorithm
implemented in C using the Intel image processing library.

\begin{thebibliography}{10}
\bibitem{AMD}Advanced Micro Devices, 3DNow! Technology Manual, 1999. 
\bibitem{Aho}Aho, A.V., Ganapathi, M, TJiang S.W.K., Code Generation Using Tree Matching
and Dynamic Programming, ACM Trans, Programming Languages and Systems 11, no.4,
1989, pp.491-516. 
\bibitem{Burke}Burke, Chris, J User Manual, ISI, 1995.
\bibitem{Cattel80}Cattell R. G. G., Automatic derivation of code generators from machine descriptions,
ACM Transactions on Programming Languages and Systems, 2(2), pp. 173-190, April
1980. 
\bibitem{Chaiten}Chaitin. G., Elegant Lisp Programs, in The Limits of Mathematics, pp. 29-56,
Springer, 1997.
\bibitem{Cheong97}Cheong, G., and Lam, M., An Optimizer for Multimedia Instruction Sets, 2nd SUIF
Workshop, Stanford University, August 1997. 
\bibitem{Cockshott00}Cockshott, Paul, Direct Compilation of High Level Languages for Multi-media
Instruction-sets, Department of Computer Science, University of Glasgow, Nov
2000.
\bibitem{graham80}Susan L. Graham, Table Driven Code Generation, IEEE Computer, Vol 13, No. 8,
August 1980, pp 25..37. 
\bibitem{IA32 }Intel, Intel Architecture Software Developers Manual Volumes 1 and 2, 1999. 
\bibitem{Intel00}Intel, Willamette Processor Software Developer's Guide, February, 2000. 
\bibitem{ISO90}ISO, Extended Pascal ISO 10206:1990, 1991.
\bibitem{Iverson62}Iverson K. E., A Programming Language, John Wiley \& Sons, Inc., New York (1962),
p. 16. 
\bibitem{Iversion80}Iverson, K. E. . Notation as a tool of thought. Communications of the ACM, 23(8),
444-465, 1980.
\bibitem{Jmanual}Iverson K. E, A personal View of APL, IBM Systems Journal, Vol 30, No 4, 1991. 
\bibitem{Jintro}Iverson, Kenneth E., J Introduction and Dictionary, Iverson Software Inc. (ISI),
Toronto, Ontario, 1995.~
\bibitem{Wirth}Jensen K., and Wirth N., Pascal User Manual and Report, Springer, 1978.
\bibitem{Krall00}Krall, A., and Lelait, S., Compilation Techniques for Multimedia Processors,
International Journal of Parallel Programming, Vol. 28, No. 4, pp 347-361, 2000. 
\bibitem{Leupers99}Leupers, R., Compiler Optimization for Media Processors, EMMSEC 99/Sweden 1999. 
\bibitem{Metcalf96}Metcalf, M., and Reid., J., The F Programming Language, OUP, 1996. 
\bibitem{Peleg97}Peleg, A., Wilke S., Weiser U., Intel MMX for Multimedia PCs, Comm. ACM, vol
40, no. 1 1997. 
\bibitem{Sreraman00}Srereman, N., and Govindarajan, G., A Vectorizing Compiler for Multimedia Extensions,
International Journal of Parallel Programming, Vol. 28, No. 4, pp 363-400, 2000. \end{thebibliography}

\end{document}
