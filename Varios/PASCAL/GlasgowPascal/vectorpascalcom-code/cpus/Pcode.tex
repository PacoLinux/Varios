\documentclass[10pt,a4paper]{article}
\usepackage[latin1]{inputenc}
\usepackage{amsmath}
\usepackage{amsfonts}
\usepackage{amssymb}
\title{Glasgow        p-code        abstract        machine}
\begin{document}
\maketitle
\input{pcode.m4}
\section{Example}
Consider        the        following        Pascal        function
\begin{verbatim}
procedure one;
begin
   a:=a+1;
   if a = 1 then two(a)
end;
\end{verbatim}
The function once it has passed through the vpc compiler with the -cpuPcode flag enabled generates the following
annotated  
assembler 
\begin{verbatim}
#procedure        one;
        label1294b80c2a2a2e:
        #	        one
        enter        spaceforonel107-4*1,1
#begin
#    a:=a+1;
        ;        #33
        vpush        ebp
        loadlit                                                8
        add        
        loadl
        vpush        ebp
        loadlit                                                8
        add        
        loadl
        loadl
        loadlit                                                1
        add        
        vstorel        
#    if        a        =        1        then        two(a)
        vpush        ebp
        loadlit                                                8
        add        
        loadl
        loadl
        loadlit                                                1
        eq
        iftrue        label1294b80c2a7a46
        goto                label1294b80c2a7a48
        label1294b80c2a7a46:
        vpush        ebp
        loadlit                                                8
        add        
        mloadl
        call        label1294b80c2a2a32
        incsp        4
        label1294b80c2a7a48:
        label1294b80c2a3a3e:
#end;
        spaceforonel107        =        4
        onel107exit:
	leave
        ret
\end{verbatim}

\end{document}