%% LyX 1.2 created this file.  For more info, see http://www.lyx.org/.
%% Do not edit unless you really know what you are doing.
\documentclass[12pt,english]{article}
\usepackage[T1]{fontenc}
\usepackage[latin1]{inputenc}

\makeatletter

%%%%%%%%%%%%%%%%%%%%%%%%%%%%%% LyX specific LaTeX commands.
\providecommand{\LyX}{L\kern-.1667em\lower.25em\hbox{Y}\kern-.125emX\@}

%%%%%%%%%%%%%%%%%%%%%%%%%%%%%% User specified LaTeX commands.
\usepackage{calc}
\usepackage{hyperref}\usepackage{setspace}
\usepackage{multicol}
\usepackage[normalem]{ulem}
\usepackage{color}
 
\setlength{\oddsidemargin}{1.2500in-1in}
\setlength{\textwidth}{\paperwidth - 1.2500in-1.2500in}



\usepackage{babel}
\makeatother
\begin{document}
\begin{flushleft}\textbf{\large A.7 ADDPS: SSE Packed Single-FP ADD}\end{flushleft}{\large \par}

\begin{flushleft}{\footnotesize ADDPS xmmreg,r/m128 ; 0F 58 /r {[}KATMAI,SSE{]}}\end{flushleft}{\footnotesize \par}

\begin{flushleft}ADDPS treats both operands as vectors of four 32-bit
floating-point numbers and adds each number in the source operand
to the corresponding number in the destination register. \end{flushleft}

\begin{flushleft}\textbf{\large A.8 ADDSS: SSE Scalar Single-FP ADD}\end{flushleft}{\large \par}

\begin{flushleft}{\footnotesize ADDSS xmmreg,xmmreg/mem32 ; F3 0F
58 /r {[}KATMAI,SSE{]}}\end{flushleft}{\footnotesize \par}

\begin{flushleft}ADDSS adds the 32-bit floating-point number in the
lowest 4 bytes of the source operand to the floating-point number
in the lowest quarter of the destination register. \end{flushleft}

\begin{flushleft}\textbf{\large A.10 ANDNPS: SSE Bitwise Logical AND
NOT}\end{flushleft}{\large \par}

\begin{flushleft}{\footnotesize ANDNPS xmmreg,r/m128 ; 0F 55 /r {[}KATMAI,SSE{]}}\end{flushleft}{\footnotesize \par}

\begin{flushleft}ANDNPS performs a bitwise AND operation on the source
operand and the complement of the destination register, and stores
the result in the destination register. \end{flushleft}

\begin{flushleft}\textbf{\large A.11 ANDPS: SSE Bitwise Logical AND}\end{flushleft}{\large \par}

\begin{flushleft}{\footnotesize ANDPS xmmreg,r/m128 ; 0F 54 /r {[}KATMAI,SSE{]}}\end{flushleft}{\footnotesize \par}

\begin{flushleft}ANDPS performs a bitwise AND operation on the source
operand and the destination register, and stores the result in the
destination register. \end{flushleft}

\begin{flushleft}\textbf{\large A.23 CMPccPS: SSE Packed Single-FP
Compare}\end{flushleft}{\large \par}

\begin{flushleft}{\footnotesize CMPPS xmmreg,r/m128,imm8 ; 0F C2 /r
ib {[}KATMAI,SSE{]}}\end{flushleft}{\footnotesize \par}

\begin{flushleft}{\footnotesize CMPEQPS xmmreg,r/m128 ; 0F C2 /r 00
{[}KATMAI,SSE{]} }\end{flushleft}{\footnotesize \par}

\begin{flushleft}{\footnotesize CMPLEPS xmmreg,r/m128 ; 0F C2 /r 02
{[}KATMAI,SSE{]} }\end{flushleft}{\footnotesize \par}

\begin{flushleft}{\footnotesize CMPLTPS xmmreg,r/m128 ; 0F C2 /r 01
{[}KATMAI,SSE{]} }\end{flushleft}{\footnotesize \par}

\begin{flushleft}{\footnotesize CMPNEQPS xmmreg,r/m128 ; 0F C2 /r
04 {[}KATMAI,SSE{]} }\end{flushleft}{\footnotesize \par}

\begin{flushleft}{\footnotesize CMPNLEPS xmmreg,r/m128 ; 0F C2 /r
06 {[}KATMAI,SSE{]} }\end{flushleft}{\footnotesize \par}

\begin{flushleft}{\footnotesize CMPNLTPS xmmreg,r/m128 ; 0F C2 /r
05 {[}KATMAI,SSE{]} }\end{flushleft}{\footnotesize \par}

\begin{flushleft}{\footnotesize CMPORDPS xmmreg,r/m128 ; 0F C2 /r
07 {[}KATMAI,SSE{]} }\end{flushleft}{\footnotesize \par}

\begin{flushleft}{\footnotesize CMPUNORDPS xmmreg,r/m128 ; 0F C2 /r
03 {[}KATMAI,SSE{]}}\end{flushleft}{\footnotesize \par}

\begin{flushleft}CMPPS treats both operands as vectors of four 32-bit
floating-point numbers. For each pair of such numbers CMPPS produces
an all 1 32-bit mask or an all 0 32-bit mask, using the comparison
specified by imm8, and puts this mask in the corresponding location
in the destination register. The allowed values of imm8 are 0-7, which
correspond to the eight pseudo-ops shown above. \end{flushleft}

\begin{flushleft}\textbf{\large A.25 CMPccSS: SSE Scalar Single-FP
Compare}\end{flushleft}{\large \par}

\begin{flushleft}{\footnotesize CMPSS xmmreg,xmmreg/mem32,imm8; F3
0F C2 /r ib {[}KATMAI,SSE{]}}\end{flushleft}{\footnotesize \par}

\begin{flushleft}{\footnotesize CMPEQSS xmmreg,xmmreg/mem32 ; F3 0F
C2 /r 00 {[}KATMAI,SSE{]} }\end{flushleft}{\footnotesize \par}

\begin{flushleft}{\footnotesize CMPLESS xmmreg,xmmreg/mem32 ; F3 0F
C2 /r 02 {[}KATMAI,SSE{]} }\end{flushleft}{\footnotesize \par}

\begin{flushleft}{\footnotesize CMPLTSS xmmreg,xmmreg/mem32 ; F3 0F
C2 /r 01 {[}KATMAI,SSE{]} }\end{flushleft}{\footnotesize \par}

\begin{flushleft}{\footnotesize CMPNEQSS xmmreg,xmmreg/mem32 ; F3
0F C2 /r 04 {[}KATMAI,SSE{]} }\end{flushleft}{\footnotesize \par}

\begin{flushleft}{\footnotesize CMPNLESS xmmreg,xmmreg/mem32 ; F3
0F C2 /r 06 {[}KATMAI,SSE{]} }\end{flushleft}{\footnotesize \par}

\begin{flushleft}{\footnotesize CMPNLTSS xmmreg,xmmreg/mem32 ; F3
0F C2 /r 05 {[}KATMAI,SSE{]} }\end{flushleft}{\footnotesize \par}

\begin{flushleft}{\footnotesize CMPORDSS xmmreg,xmmreg/mem32 ; F3
0F C2 /r 07 {[}KATMAI,SSE{]} }\end{flushleft}{\footnotesize \par}

\begin{flushleft}{\footnotesize CMPUNORDSS xmmreg,xmmreg/mem32; F3
0F C2 /r 03 {[}KATMAI,SSE{]}}\end{flushleft}{\footnotesize \par}

\begin{flushleft}CMPSS is the same as CMPPS except that it compares
only the first pair of 32-bit floating point numbers. \end{flushleft}

\begin{flushleft}\textbf{\large A.28 COMISS: SSE Scalar Compare and
Set EFLAGS}\end{flushleft}{\large \par}

\begin{flushleft}{\footnotesize COMISS xmmreg,xmmreg/mem32 ; 0F 2F
/r {[}KATMAI,SSE{]}}\end{flushleft}{\footnotesize \par}

\begin{flushleft}COMISS compares the 32-bit floating-point numbers
in the lowest part of the two operands, and sets the CPU flags appropriately.
COMISS differs from UCOMISS in that it signals an invalid numeric
exeception if an operand is an sNaN or a qNaN, whereas UCOMISS does
so only if an operand is an sNaN. \end{flushleft}

\begin{flushleft}\textbf{\large A.30 CVTPI2PS: SSE Packed Integer
to Floating-Point Conversion}\end{flushleft}{\large \par}

\begin{flushleft}{\footnotesize CVTPI2PS xmmreg,r/m64 ; 0F 2A /r {[}KATMAI,SSE{]}}\end{flushleft}{\footnotesize \par}

\begin{flushleft}CVTPI2PS considers the source operand as a pair of
signed 32-bit integers and converts them to 32-bit floating-point
numbers, storing the result in the lower half of the destination register. \end{flushleft}

\begin{flushleft}\textbf{\large A.31 CVTPS2PI, CVTTPS2PI: SSE Packed
Floating-Point to Integer Conversion}\end{flushleft}{\large \par}

\begin{flushleft}{\footnotesize CVTPS2PI mmxreg,xmmreg/mem64 ; 0F
2D /r {[}KATMAI,SSE{]} }\end{flushleft}{\footnotesize \par}

\begin{flushleft}{\footnotesize CVTTPS2PI mmxreg,xmmreg/mem64 ; 0F
2C /r {[}KATMAI,SSE{]}}\end{flushleft}{\footnotesize \par}

\begin{flushleft}These instructions consider the source operand as
a pair of 32-bit floating-point numbers and convert them to signed
32-bit integers, storing the result in the destination register. Note
that if the source operand is a register then only its lower half
is used. If the conversion is inexact, then CVTTPS2PI truncates, whereas
CVTPS2PI rounds according to the MXCSR. \end{flushleft}

\begin{flushleft}\textbf{\large A.32 CVTSI2SS: SSE Scalar Integer
to Floating-Point Conversion}\end{flushleft}{\large \par}

\begin{flushleft}{\footnotesize CVTSI2SS xmmreg,r/m32 ; F3 0F 2A /r
{[}KATMAI,SSE{]}}\end{flushleft}{\footnotesize \par}

\begin{flushleft}CVTSI2SS considers the source operand as a 32-bit
signed integer, and converts it to a 32-bit float, storing the result
in the lowest quarter of the destination register. \end{flushleft}

\begin{flushleft}\textbf{\large A.33 CVTSS2SI, CVTTSS2SI: SSE Scalar
Floating-Point to Integer Conversion}\end{flushleft}{\large \par}

\begin{flushleft}{\footnotesize CVTSS2SI reg32,xmmreg/mem32 ; F3 0F
2D /r {[}KATMAI,SSE{]} }\end{flushleft}{\footnotesize \par}

\begin{flushleft}{\footnotesize CVTTSS2SI reg32,xmmreg/mem32 ; F3
0F 2C /r {[}KATMAI,SSE{]}}\end{flushleft}{\footnotesize \par}

\begin{flushleft}These instructions consider the source operand as
a 32-bit floating-point number and convert it to a signed 32-bit integer,
storing the result in the destination register. Note that if the source
operand is a register then only its lowest quarter is used. If the
conversion is inexact, then CVTTSS2SI truncates, whereas CVTSS2SI
rounds according to the MXCSR. \end{flushleft}

\begin{flushleft}\textbf{\large A.37 DIVPS: Packed Single-FP Divide}\end{flushleft}{\large \par}

\begin{flushleft}{\footnotesize DIVPS xmmreg,r/m128 ; 0F 5E /r {[}KATMAI,SSE{]}}\end{flushleft}{\footnotesize \par}

\begin{flushleft}DIVPS considers both operands as vectors of four
32-bit floating-point numbers and divides each such number in the
destination register by the corresponding number in the source operand. \end{flushleft}

\begin{flushleft}\textbf{\large A.38 DIVSS: Scalar Single-FP Divide}\end{flushleft}{\large \par}

\begin{flushleft}{\footnotesize DIVSS xmmreg,xmmreg/mem32 ; F3 0F
5E /r {[}KATMAI,SSE{]}}\end{flushleft}{\footnotesize \par}

\begin{flushleft}c\{DIVSS\} divides the 32-bit floating-point number
in the lowest quarter of the destination register by the corresponding
number in the source operand. \end{flushleft}

\begin{flushleft}\textbf{\large A.39 EMMS: Empty MMX State}\end{flushleft}{\large \par}

\begin{flushleft}{\footnotesize EMMS ; 0F 77 {[}PENT,MMX{]}}\end{flushleft}{\footnotesize \par}

\begin{flushleft}EMMS sets the FPU tag word (marking which floating-point
registers are available) to all ones, meaning all registers are available
for the FPU to use. It should be used after executing MMX instructions
and before executing any subsequent floating-point operations. \end{flushleft}

\begin{flushleft}\textbf{\large A.53 FEMMS: Fast EMMS}\end{flushleft}{\large \par}

\begin{flushleft}{\footnotesize FEMMS ; 0F 0E {[}3DNOW{]}}\end{flushleft}{\footnotesize \par}

\begin{flushleft}FEMMS is like EMMS except that it is faster and leaves
the contents of the MMX / floating-point registers undefined. \end{flushleft}

\begin{flushleft}\textbf{\large A.88 FXRSTOR: Restore FP, MMX and
SSE States}\end{flushleft}{\large \par}

\begin{flushleft}{\footnotesize FXRSTOR m512byte ; 0F AE /1 {[}P6,SSE,FPU{]}}\end{flushleft}{\footnotesize \par}

\begin{flushleft}FXRSTOR reloads the FP and MMX state, and the SSE
state (environment and registers), from the memory area defined by
m512byte. This data should have been written by a previous FXSAVE. \end{flushleft}

\begin{flushleft}\textbf{\large A.89 FXSAVE: Store FP and MMX State
and Streaming SIMD}\end{flushleft}{\large \par}

\begin{flushleft}{\footnotesize FXSAVE m512byte ; 0F AE /0 {[}P6,SSE,FPU{]}}\end{flushleft}{\footnotesize \par}

\begin{flushleft}FXSAVE writes the current FP and MMX state, and SSE
state (environment and registers), to the specified destination defined
by m512byte. It does this without checking for pending unmasked floating-point
exceptions (similar to the operation of FNSAVE). Unlike the FSAVE/FNSAVE
instructions, the processor retains the contents of the FP and MMX
state and Streaming SIMD Extension state in the processor after the
state has been saved. This instruction has been optimized to maximize
floating-point save performance. \end{flushleft}

\begin{flushleft}\textbf{\large A}\end{flushleft}{\large \par}

\begin{flushleft}\textbf{\large A.111 LDMXCSR: SSE Load MXCSR}\end{flushleft}{\large \par}

\begin{flushleft}{\footnotesize LDMXCSR mem32 ; 0F AE /2 {[}KATMAI,SSE{]}}\end{flushleft}{\footnotesize \par}

\begin{flushleft}LDMXCSR loads a 32-bit value out of memory and stores
it into the MXCSR (the SSE control/status register). \end{flushleft}

\begin{flushleft}\textbf{\large A.121 MASKMOVQ: Byte Mask Write}\end{flushleft}{\large \par}

\begin{flushleft}{\footnotesize MASKMOVQ mmxreg,mmxreg ; 0F F7 /r
{[}KATMAI{]}}\end{flushleft}{\footnotesize \par}

\begin{flushleft}MASKMOVQ uses the most significant bit in each byte
of the second MMX resister to selectively write (0 = no write, 1 =
write) each byte of the first MMX register to the 64-bit memory location
DS:DI or DS:EDI (depending on the addressing mode). \end{flushleft}

\begin{flushleft}\textbf{\large A.122 MAXPS: SSE Packed Single-FP
Maximum}\end{flushleft}{\large \par}

\begin{flushleft}{\footnotesize MAXPS xmmreg,r/m128 ; 0F 5F /r {[}KATMAI,SSE{]}}\end{flushleft}{\footnotesize \par}

\begin{flushleft}MAXPS considers its operands as vectors of four 32-bit
floating-point numbers, and for each pair it stores the maximum of
the the two in the corresponding quarter of the destination register. \end{flushleft}

\begin{flushleft}\textbf{\large A.123 MAXSS: SSE Scalar Single-FP
Maximum}\end{flushleft}{\large \par}

\begin{flushleft}{\footnotesize MAXSS xmmreg,r/m128 ; F3 0F 5F /r
{[}KATMAI,SSE{]}}\end{flushleft}{\footnotesize \par}

\begin{flushleft}MAXSS determines the maximum 32-bit floating-point
number from the lowest quarter of both operands, and places this in
the lowest quarter of the destination register. \end{flushleft}

\begin{flushleft}\textbf{\large A.124 MINPS: SSE Packed Single-FP
Minimum}\end{flushleft}{\large \par}

\begin{flushleft}{\footnotesize MINPS xmmreg,r/m128 ; 0F 5D /r {[}KATMAI,SSE{]}}\end{flushleft}{\footnotesize \par}

\begin{flushleft}MINPS considers its operands as vectors of four 32-bit
floating-point numbers, and for each pair it stores the minimum of
the the two in the corresponding quarter of the destination register. \end{flushleft}

\begin{flushleft}\textbf{\large A.125 MINSS: SSE Scalar Single-FP
Maximum}\end{flushleft}{\large \par}

\begin{flushleft}{\footnotesize MINSS xmmreg,r/m128 ; F3 0F 5D /r
{[}KATMAI,SSE{]}}\end{flushleft}{\footnotesize \par}

\begin{flushleft}MINSS determines the minimum 32-bit floating-point
number from the lowest quarter of both operands, and places this in
the lowest quarter of the destination register. \end{flushleft}

\begin{flushleft}\textbf{\large A.127 MOVAPS: Move Aligned Four Packed
Single-FP}\end{flushleft}{\large \par}

\begin{flushleft}{\footnotesize MOVAPS xmmreg,r/m128 ; 0F 28 /r {[}KATMAI,SSE{]} }\end{flushleft}{\footnotesize \par}

\begin{flushleft}{\footnotesize MOVAPS r/m128,xmmreg ; 0F 29 /r {[}KATMAI,SSE{]}}\end{flushleft}{\footnotesize \par}

\begin{flushleft}MOVAPS copies 16 bytes from the source operand to
the destination operand. If one of the operands is a memory location
it must be aligned on a 16-byte boundary (otherwise use MOVUPS). \end{flushleft}

\begin{flushleft}\textbf{\large A.128 MOVD: Move Doubleword to/from
MMX Register}\end{flushleft}{\large \par}

\begin{flushleft}{\footnotesize MOVD mmxreg,r/m32 ; 0F 6E /r {[}PENT,MMX{]} }\end{flushleft}{\footnotesize \par}

\begin{flushleft}{\footnotesize MOVD r/m32,mmxreg ; 0F 7E /r {[}PENT,MMX{]}}\end{flushleft}{\footnotesize \par}

\begin{flushleft}MOVD copies 32 bits from its source (second) operand
into its destination (first) operand. When the destination is a 64-bit
MMX register, the top 32 bits are set to zero. \end{flushleft}

\begin{flushleft}\textbf{\large A.129 MOVHLPS: SSE Move High to Low}\end{flushleft}{\large \par}

\begin{flushleft}{\footnotesize MOVHLPS xmmreg,xmmreg ; OF 12 /r {[}KATMAI,SSE{]}}\end{flushleft}{\footnotesize \par}

\begin{flushleft}MOVHLPS moves 8 bytes from the upper half of the
source register to the lower half of the destination register. \end{flushleft}

\begin{flushleft}\textbf{\large A.130 MOVHPS: SSE Move High}\end{flushleft}{\large \par}

\begin{flushleft}{\footnotesize MOVHPS xmmreg,mem64 ; 0F 16 /r {[}KATMAI,SSE{]} }\end{flushleft}{\footnotesize \par}

\begin{flushleft}{\footnotesize MOVHPS mem64,xmmreg ; 0F 17 /r {[}KATMAI,SSE{]} }\end{flushleft}{\footnotesize \par}

\begin{flushleft}{\footnotesize MOVHPS xmmreg,xmmreg ; 0F 16 /r {[}KATMAI,SSE{]}}\end{flushleft}{\footnotesize \par}

\begin{flushleft}MOVHPS xmmreg,mem64 moves 8 bytes from mem64 to the
upper half of xmmreg. \end{flushleft}

\begin{flushleft}MOVHPS mem64,xmmreg moves 8 bytes from the upper
half of xmmreg to mem64. \end{flushleft}

\begin{flushleft}MOVHPS xmmreg,xmmreg is simply a synonym for MOVLHPS
xmmreg,xmmreg. \end{flushleft}

\begin{flushleft}\textbf{\large A.131 MOVLHPS: SSE Move Low to High}\end{flushleft}{\large \par}

\begin{flushleft}{\footnotesize MOVLHPS xmmreg,xmmreg ; 0F 16 /r {[}KATMAI,SSE{]}}\end{flushleft}{\footnotesize \par}

\begin{flushleft}MOVLHPS moves 8 bytes from the lower half of the
source register to the upper half of the destination register. \end{flushleft}

\begin{flushleft}\textbf{\large A.132 MOVLPS: SSE Move Low}\end{flushleft}{\large \par}

\begin{flushleft}{\footnotesize MOVLPS xmmreg,mem64 ; 0F 12 /r {[}KATMAI,SSE{]} }\end{flushleft}{\footnotesize \par}

\begin{flushleft}{\footnotesize MOVLPS mem64,xmmreg ; 0F 13 /r {[}KATMAI,SSE{]} }\end{flushleft}{\footnotesize \par}

\begin{flushleft}{\footnotesize MOVLPS xmmreg,xmmreg ; 0F 12 /r {[}KATMAI,SSE{]}}\end{flushleft}{\footnotesize \par}

\begin{flushleft}MOVLPS xmmreg,mem64 moves 8 bytes from mem64 to the
lower half of xmmreg. \end{flushleft}

\begin{flushleft}MOVLPS mem64,xmmreg moves 8 bytes from the lower
half of xmmreg to mem64. \end{flushleft}

\begin{flushleft}MOVLPS xmmreg,xmmreg is simply a synonym for MOVHLPS
xmmreg,xmmreg. \end{flushleft}

\begin{flushleft}\textbf{\large A.133 MOVMSKPS: Move Mask To Integer}\end{flushleft}{\large \par}

\begin{flushleft}{\footnotesize MOVMSKPS reg32,xmmreg ; 0F 50 /r {[}KATMAI,SSE{]}}\end{flushleft}{\footnotesize \par}

\begin{flushleft}MOVMSKPS forms a 4-bit mask from the most significant
bit of each of the four 32-bit numbers in the source register, and
stores this mask in the destination register. \end{flushleft}

\begin{flushleft}\textbf{\large A.134 MOVNTPS: Move Aligned Four Packed
Single-FP Non Temporal}\end{flushleft}{\large \par}

\begin{flushleft}{\footnotesize MOVNTPS mem128,xmmreg ; 0F 2B /r {[}KATMAI,SSE{]}}\end{flushleft}{\footnotesize \par}

\begin{flushleft}MOVNTPS copies the contents of the XMM register into
the given memory location, doing so in such a way as to minimize cache
pollution. The memory location must be 16-byte aligned. \end{flushleft}

\begin{flushleft}\textbf{\large A.135 MOVNTQ: Move 64 Bits Non Temporal}\end{flushleft}{\large \par}

\begin{flushleft}{\footnotesize MOVNTQ mem64,mmxreg ; 0F E7 /r {[}KATMAI{]}}\end{flushleft}{\footnotesize \par}

\begin{flushleft}MOVNTPS copies the contents of the MMX register into
the given memory location, doing so in such a way as to minimize cache
pollution. \end{flushleft}

\begin{flushleft}\textbf{\large A.136 MOVQ: Move Quadword to/from
MMX Register}\end{flushleft}{\large \par}

\begin{flushleft}{\footnotesize MOVQ mmxreg,r/m64 ; 0F 6F /r {[}PENT,MMX{]} }\end{flushleft}{\footnotesize \par}

\begin{flushleft}{\footnotesize MOVQ r/m64,mmxreg ; 0F 7F /r {[}PENT,MMX{]}}\end{flushleft}{\footnotesize \par}

\begin{flushleft}MOVQ copies 64 bits from its source (second) operand
into its destination (first) operand. \end{flushleft}

\begin{flushleft}\textbf{\large A.138 MOVSS: Move Scalar Single-FP}\end{flushleft}{\large \par}

\begin{flushleft}{\footnotesize MOVSS xmmreg,xmmreg/mem32 ; F3 0F
10 /r {[}KATMAI,SSE{]} }\end{flushleft}{\footnotesize \par}

\begin{flushleft}{\footnotesize MOVSS xmmreg/mem32,xmmreg ; F3 0F
11 /r {[}KATMAI,SSE{]}}\end{flushleft}{\footnotesize \par}

\begin{flushleft}MOVSS copies the lower 4 bytes of the source operand
to the lower 4 bytes of the destination operand. \end{flushleft}

\begin{flushleft}\textbf{\large A.140 MOVUPS: Move Unaligned Four
Packed Single-FP}\end{flushleft}{\large \par}

\begin{flushleft}{\footnotesize MOVUPS xmmreg,r/m128 ; 0F 10 /r {[}KATMAI,SSE{]} }\end{flushleft}{\footnotesize \par}

\begin{flushleft}{\footnotesize MOVUPS r/m128,xmmreg ; 0F 11 /r {[}KATMAI,SSE{]}}\end{flushleft}{\footnotesize \par}

\begin{flushleft}MOVUPS copies 16 bytes from the source operand to
the destination operand. In contrast to MOVAPS, no assumption is made
about alignment. \end{flushleft}

\begin{flushleft}\textbf{\large A.143 MULSS: Scalar Single-FP Multiply}\end{flushleft}{\large \par}

\begin{flushleft}{\footnotesize MULSS xmmreg,xmmreg/mem32 ; F3 0F
59 /r {[}KATMAI,SSE{]}}\end{flushleft}{\footnotesize \par}

\begin{flushleft}MULSS multiplies the first of the four 32-bit floating-point
numbers in the destination register by the corresponding number in
the source operand. \end{flushleft}

\begin{flushleft}\textbf{\large A.147 ORPS: SSE Bitwise Logical OR}\end{flushleft}{\large \par}

\begin{flushleft}{\footnotesize ORPS xmmreg,r/m128 ; 0F 56 /r {[}KATMAI,SSE{]}}\end{flushleft}{\footnotesize \par}

\begin{flushleft}ORPS performs a bitwise OR operation on the source
operand and the destination register, and stores the result in the
destination register. \end{flushleft}

\begin{flushleft}\textbf{\large A.150 PACKSSDW, PACKSSWB, PACKUSWB:
Pack Data}\end{flushleft}{\large \par}

\begin{flushleft}{\footnotesize PACKSSDW mmxreg,r/m64 ; 0F 6B /r {[}PENT,MMX{]} }\end{flushleft}{\footnotesize \par}

\begin{flushleft}{\footnotesize PACKSSWB mmxreg,r/m64 ; 0F 63 /r {[}PENT,MMX{]} }\end{flushleft}{\footnotesize \par}

\begin{flushleft}{\footnotesize PACKUSWB mmxreg,r/m64 ; 0F 67 /r {[}PENT,MMX{]}}\end{flushleft}{\footnotesize \par}

\begin{flushleft}All these instructions start by forming a notional
128-bit word by placing the source (second) operand on the left of
the destination (first) operand. PACKSSDW then splits this 128-bit
word into four doublewords, converts each to a word, and loads them
side by side into the destination register; PACKSSWB and PACKUSWB
both split the 128-bit word into eight words, converts each to a byte,
and loads those side by side into the destination register. \end{flushleft}

\begin{flushleft}PACKSSDW and PACKSSWB perform signed saturation when
reducing the length of numbers: if the number is too large to fit
into the reduced space, they replace it by the largest signed number
(7FFFh or 7Fh) that will fit, and if it is too small then they replace
it by the smallest signed number (8000h or 80h) that will fit. PACKUSWB
performs unsigned saturation: it treats its input as unsigned, and
replaces it by the largest unsigned number that will fit. \end{flushleft}

\begin{flushleft}\textbf{\large A.151 PADDxx: MMX Packed Addition}\end{flushleft}{\large \par}

\begin{flushleft}{\footnotesize PADDB mmxreg,r/m64 ; 0F FC /r {[}PENT,MMX{]} }\end{flushleft}{\footnotesize \par}

\begin{flushleft}{\footnotesize PADDW mmxreg,r/m64 ; 0F FD /r {[}PENT,MMX{]} }\end{flushleft}{\footnotesize \par}

\begin{flushleft}{\footnotesize PADDD mmxreg,r/m64 ; 0F FE /r {[}PENT,MMX{]}}\end{flushleft}{\footnotesize \par}

\begin{flushleft}{\footnotesize PADDSB mmxreg,r/m64 ; 0F EC /r {[}PENT,MMX{]} }\end{flushleft}{\footnotesize \par}

\begin{flushleft}{\footnotesize PADDSW mmxreg,r/m64 ; 0F ED /r {[}PENT,MMX{]}}\end{flushleft}{\footnotesize \par}

\begin{flushleft}{\footnotesize PADDUSB mmxreg,r/m64 ; 0F DC /r {[}PENT,MMX{]} }\end{flushleft}{\footnotesize \par}

\begin{flushleft}{\footnotesize PADDUSW mmxreg,r/m64 ; 0F DD /r {[}PENT,MMX{]}}\end{flushleft}{\footnotesize \par}

\begin{flushleft}PADDxx all perform packed addition between their
two 64-bit operands, storing the result in the destination (first)
operand. The PADDxB forms treat the 64-bit operands as vectors of
eight bytes, and add each byte individually; PADDxW treat the operands
as vectors of four words; and PADDD treats its operands as vectors
of two doublewords. \end{flushleft}

\begin{flushleft}PADDSB and PADDSW perform signed saturation on the
sum of each pair of bytes or words: if the result of an addition is
too large or too small to fit into a signed byte or word result, it
is clipped (saturated) to the largest or smallest value which will
fit. PADDUSB and PADDUSW similarly perform unsigned saturation, clipping
to 0FFh or 0FFFFh if the result is larger than that. \end{flushleft}

\begin{flushleft}\textbf{\large A.152 PADDSIW: MMX Packed Addition
to Implicit Destination}\end{flushleft}{\large \par}

\begin{flushleft}{\footnotesize PADDSIW mmxreg,r/m64 ; 0F 51 /r {[}CYRIX,MMX{]}}\end{flushleft}{\footnotesize \par}

\begin{flushleft}PADDSIW, specific to the Cyrix extensions to the
MMX instruction set, performs the same function as PADDSW, except
that the result is not placed in the register specified by the first
operand, but instead in the register whose number differs from the
first operand only in the last bit. So PADDSIW MM0,MM2 would put the
result in MM1, but PADDSIW MM1,MM2 would put the result in MM0. \end{flushleft}

\begin{flushleft}\textbf{\large A.153 PAND, PANDN: MMX Bitwise AND
and AND-NOT}\end{flushleft}{\large \par}

\begin{flushleft}{\footnotesize PAND mmxreg,r/m64 ; 0F DB /r {[}PENT,MMX{]} }\end{flushleft}{\footnotesize \par}

\begin{flushleft}{\footnotesize PANDN mmxreg,r/m64 ; 0F DF /r {[}PENT,MMX{]}}\end{flushleft}{\footnotesize \par}

\begin{flushleft}PAND performs a bitwise AND operation between its
two operands (i.e. each bit of the result is 1 if and only if the
corresponding bits of the two inputs were both 1), and stores the
result in the destination (first) operand. \end{flushleft}

\begin{flushleft}PANDN performs the same operation, but performs a
one's complement operation on the destination (first) operand first. \end{flushleft}

\begin{flushleft}\textbf{\large A.154 PAVEB: MMX Packed Average}\end{flushleft}{\large \par}

\begin{flushleft}{\footnotesize PAVEB mmxreg,r/m64 ; 0F 50 /r {[}CYRIX,MMX{]}}\end{flushleft}{\footnotesize \par}

\begin{flushleft}PAVEB, specific to the Cyrix MMX extensions, treats
its two operands as vectors of eight unsigned bytes, and calculates
the average of the corresponding bytes in the operands. The resulting
vector of eight averages is stored in the first operand. \end{flushleft}

\begin{flushleft}\textbf{\large A.155 PAVGB, PAVGW: Packed Average}\end{flushleft}{\large \par}

\begin{flushleft}{\footnotesize PAVGB mmxreg,r/m64 ; 0F E0 /r {[}KATMAI{]} }\end{flushleft}{\footnotesize \par}

\begin{flushleft}{\footnotesize PAVGW mmxreg,r/m64 ; 0F E3 /r {[}KATMAI{]}}\end{flushleft}{\footnotesize \par}

\begin{flushleft}For each byte in the source register, PAVGB computes
the average of this byte and the corresponding byte in the destination
register, and stores this average in place of the byte in the source
register. \end{flushleft}

\begin{flushleft}PAVGW does the same thing, but operating on 4 pairs
of words instead of 8 pairs of bytes. \end{flushleft}

\begin{flushleft}In all cases, the values operated one are considered
to be unsigned, and the result is rounded up if it is not an integer. \end{flushleft}

\begin{flushleft}\textbf{\large A.156 PAVGUSB: Average Of Unsigned
Packed 8-bit Values}\end{flushleft}{\large \par}

\begin{flushleft}{\footnotesize PAVGUSB mmxreg,r/m64 ; 0F 0F /r BF
{[}3DNOW{]}}\end{flushleft}{\footnotesize \par}

\begin{flushleft}PAVGUSB produces the averages (rounded up) of the
eight unsigned 8-bit integer values in the source operand and the
eight corresponding unsigned 8-bit integer values in the destination
register. \end{flushleft}

\begin{flushleft}\textbf{\large A.157 PCMPxx: MMX Packed Comparison}\end{flushleft}{\large \par}

\begin{flushleft}{\footnotesize PCMPEQB mmxreg,r/m64 ; 0F 74 /r {[}PENT,MMX{]} }\end{flushleft}{\footnotesize \par}

\begin{flushleft}{\footnotesize PCMPEQW mmxreg,r/m64 ; 0F 75 /r {[}PENT,MMX{]} }\end{flushleft}{\footnotesize \par}

\begin{flushleft}{\footnotesize PCMPEQD mmxreg,r/m64 ; 0F 76 /r {[}PENT,MMX{]}}\end{flushleft}{\footnotesize \par}

\begin{flushleft}{\footnotesize PCMPGTB mmxreg,r/m64 ; 0F 64 /r {[}PENT,MMX{]} }\end{flushleft}{\footnotesize \par}

\begin{flushleft}{\footnotesize PCMPGTW mmxreg,r/m64 ; 0F 65 /r {[}PENT,MMX{]} }\end{flushleft}{\footnotesize \par}

\begin{flushleft}{\footnotesize PCMPGTD mmxreg,r/m64 ; 0F 66 /r {[}PENT,MMX{]}}\end{flushleft}{\footnotesize \par}

\begin{flushleft}The PCMPxx instructions all treat their operands
as vectors of bytes, words, or doublewords; corresponding elements
of the source and destination are compared, and the corresponding
element of the destination (first) operand is set to all zeros or
all ones depending on the result of the comparison. \end{flushleft}

\begin{flushleft}PCMPxxB treats the operands as vectors of eight bytes,
PCMPxxW treats them as vectors of four words, and PCMPxxD as two doublewords. \end{flushleft}

\begin{flushleft}PCMPEQx sets the corresponding element of the destination
operand to all ones if the two elements compared are equal; PCMPGTx
sets the destination element to all ones if the element of the first
(destination) operand is greater (treated as a signed integer) than
that of the second (source) operand. \end{flushleft}

\begin{flushleft}\textbf{\large A.158 PDISTIB: MMX Packed Distance
and Accumulate with Implied Register}\end{flushleft}{\large \par}

\begin{flushleft}{\footnotesize PDISTIB mmxreg,mem64 ; 0F 54 /r {[}CYRIX,MMX{]}}\end{flushleft}{\footnotesize \par}

\begin{flushleft}PDISTIB, specific to the Cyrix MMX extensions, treats
its two input operands as vectors of eight unsigned bytes. For each
byte position, it finds the absolute difference between the bytes
in that position in the two input operands, and adds that value to
the byte in the same position in the implied output register. The
addition is saturated to an unsigned byte in the same way as PADDUSB. \end{flushleft}

\begin{flushleft}The implied output register is found in the same
way as PADDSIW (\uline{\textcolor[rgb]{0.000,0.000,1.000}{section A.152)}}. \end{flushleft}

\begin{flushleft}Note that PDISTIB cannot take a register as its second
source operand. \end{flushleft}

\begin{flushleft}\textbf{\large A.159 PEXTRW: Extract Word}\end{flushleft}{\large \par}

\begin{flushleft}{\footnotesize PEXTRW reg32,mmxreg,imm8 ; 0F C5 /r
ib {[}KATMAI{]}}\end{flushleft}{\footnotesize \par}

\begin{flushleft}PEXTRW moves the word in the MMX register (selected
by the two least significant bits of imm8) into the lower half of
the 32-bit integer register. \end{flushleft}

\begin{flushleft}\textbf{\large A.160 PF2ID: Packed Floating-Point
To Integer Conversion}\end{flushleft}{\large \par}

\begin{flushleft}{\footnotesize PF2ID mmxreg,r/m64 ; 0F 0F /r 1D {[}3DNOW{]}}\end{flushleft}{\footnotesize \par}

\begin{flushleft}PF2ID converts two 32-bit floating point numbers
in the source operand into 32-bit signed integers in the destination
register, using truncation. \end{flushleft}

\begin{flushleft}\textbf{\large A.161 PF2IW: Packed Floating-Point
to Integer Conversion}\end{flushleft}{\large \par}

\begin{flushleft}{\footnotesize PF2IW mmxreg,r/m64 ; 0F 0F /r 1C {[}ATHLON{]}}\end{flushleft}{\footnotesize \par}

\begin{flushleft}PF2IW converts two 32-bit floating point numbers
in the source operand into 16-bit signed integers in the destination
register, using truncation and sign-extending to 32 bits. \end{flushleft}

\begin{flushleft}\textbf{\large A.162 PFACC: Floating-Point Accumulate}\end{flushleft}{\large \par}

\begin{flushleft}{\footnotesize PFACC mmxreg,r/m64 ; 0F 0F /r AE {[}3DNOW{]}}\end{flushleft}{\footnotesize \par}

\begin{flushleft}PFACC treats the source and destination operands
as pairs of 32-bit floating-point numbers. The sum of the pair in
the destination register is stored in the lower half of the destination
register, and the sum of the pair in the source operand is stored
in the upper half of the destination register. \end{flushleft}

\begin{flushleft}\textbf{\large A.163 PFADD: Packed Floating-Point
Addition}\end{flushleft}{\large \par}

\begin{flushleft}{\footnotesize PFADD mmxreg,r/m64 ; 0F 0F /r 9E {[}3DNOW{]}}\end{flushleft}{\footnotesize \par}

\begin{flushleft}PFADD adds the contents of the source operand to
the contents of the destination register, treating both as pairs of
32-bit floating-point numbers. \end{flushleft}

\begin{flushleft}\textbf{\large A.164 PFCMPEQ, PFCMPGE, PFCMPGT: Packed
Floating-Point Comparison.}\end{flushleft}{\large \par}

\begin{flushleft}{\footnotesize PFCMPEQ mmxreg,r/m64 ; 0F 0F /r B0
{[}3DNOW{]} }\end{flushleft}{\footnotesize \par}

\begin{flushleft}{\footnotesize PFCMPGE mmxreg,r/m64 ; 0F 0F /r 90
{[}3DNOW{]} }\end{flushleft}{\footnotesize \par}

\begin{flushleft}{\footnotesize PFCMPGT mmxreg,r/m64 ; 0F 0F /r A0
{[}3DNOW{]}}\end{flushleft}{\footnotesize \par}

\begin{flushleft}These instructions perform comparisons between pairs
of 32-bit floating-point numbers, storing the two results in the destination
register. PFCMPEQ stores 0xFFFFFFFF if the numbers are equal, and
0 otherwise. PFCMPGE stores 0xFFFFFFFF if the destination is greater
than or equal to the source, and 0 otherwise. PFCMPGT stores 0xFFFFFFFF
if the destination is greater than the source, and 0 otherwise. \end{flushleft}

\begin{flushleft}\textbf{\large A.165 PFMAX: Packed Floating-Point
Maximum}\end{flushleft}{\large \par}

\begin{flushleft}{\footnotesize PFMAX mmxreg,r/m64 ; 0F 0F /r A4 {[}3DNOW{]}}\end{flushleft}{\footnotesize \par}

\begin{flushleft}For each half of the destination register, sets it
equal to the maximum of itself and the corresponding half of the source
operand, treating both as 32-bit floating-point numbers. \end{flushleft}

\begin{flushleft}\textbf{\large A.166 PFMIN: Packed Floating-Point
Minimum}\end{flushleft}{\large \par}

\begin{flushleft}{\footnotesize PFMIN mmxreg,r/m64 ; 0F 0F /r 94 {[}3DNOW{]}}\end{flushleft}{\footnotesize \par}

\begin{flushleft}For each half of the destination register, sets it
equal to the minimum of itself and the corresponding half of the source
operand, treating both as 32-bit floating-point numbers. \end{flushleft}

\begin{flushleft}\textbf{\large A.167 PFMUL: Packed Floating-Point
Multiply}\end{flushleft}{\large \par}

\begin{flushleft}{\footnotesize PFMUL mmxreg,r/m64 ; 0F 0F /r B4 {[}3DNOW{]}}\end{flushleft}{\footnotesize \par}

\begin{flushleft}PFMUL multiples the contents of the destination register
by the contents of the source operand, treating both as pairs of 32-bit
floating-point numbers. \end{flushleft}

\begin{flushleft}\textbf{\large A.168 PFNACC: Packed Floating-Point
Negative Accumulate}\end{flushleft}{\large \par}

\begin{flushleft}{\footnotesize PFNACC mmxreg,r/m64 ; 0F 0F /r 8A
{[}ATHLON{]}}\end{flushleft}{\footnotesize \par}

\begin{flushleft}PFNACC works the same as PFACC, except that the difference
rather than the sum is stored, the value in the upper half in both
cases being subtracted from the value in the lower half. \end{flushleft}

\begin{flushleft}\textbf{\large A.169 PFPNACC: Packed Floating-Point
Mixed Accumulate}\end{flushleft}{\large \par}

\begin{flushleft}{\footnotesize PFPNACC mmxreg,r/m64 ; 0F 0F /r 8E
{[}ATHLON{]}}\end{flushleft}{\footnotesize \par}

\begin{flushleft}PFPNACC is a mixture of PFACC and PFNACC. The new
value of the lower half of the destination register is obtained by
subtracting the upper half from the lower half. But the new value
of the upper half of the destination register is obtained by adding
both halves of the source operand. \end{flushleft}

\begin{flushleft}\textbf{\large A.170 PFRCP: Floating-Point Reciprocal
Approximation}\end{flushleft}{\large \par}

\begin{flushleft}{\footnotesize PFRCP mmxreg,r/m64 ; 0F 0F /r 96 {[}3DNOW{]}}\end{flushleft}{\footnotesize \par}

\begin{flushleft}PFRCP calculates an approximation (accurate to 14
bits) of the reciprocal of the 32-bit floating-point number in the
lower half of the source and stores it in both halves of the destination. \end{flushleft}

\begin{flushleft}\textbf{\large A.171 PFRCPIT1: Floating-Point Reciprocal
Refinement}\end{flushleft}{\large \par}

\begin{flushleft}{\footnotesize PFRCPIT1 mmxreg,r/m64 ; 0F 0F /r A6
{[}3DNOW{]}}\end{flushleft}{\footnotesize \par}

\begin{flushleft}PFRCPIT1 performs the first step in the iterative
refinement of a reciprocal produced by PFRCP. \end{flushleft}

\begin{flushleft}\textbf{\large A.172 PFRCPIT2: Floating-Point Refinement
(Last Step)}\end{flushleft}{\large \par}

\begin{flushleft}{\footnotesize PFRCPIT2 mmxreg,r/m64 ; 0F 0F /r B6
{[}3DNOW{]}}\end{flushleft}{\footnotesize \par}

\begin{flushleft}PFRCPIT2 performs the second and final step in the
iterative refinement of a reciprocal produced by PFRCP or of a reciprocal
square root produced by PFRSQRT. \end{flushleft}

\begin{flushleft}\textbf{\large A.173 PFRSQIT1: Floating-Point Reciprocal
Square-Root Refinement}\end{flushleft}{\large \par}

\begin{flushleft}{\footnotesize PFRSQIT1 mmxreg,r/m64 ; 0F 0F /r A7
{[}3DNOW{]}}\end{flushleft}{\footnotesize \par}

\begin{flushleft}PFRSQIT1 performs the first step in the iterative
refinement of a reciprocal square root produced by PFRSQRT. \end{flushleft}

\begin{flushleft}\textbf{\large A.174 PFRSQRT: Floating-Point Reciprocal
Square-Root Approximation}\end{flushleft}{\large \par}

\begin{flushleft}{\footnotesize PFRSQRT mmxreg,r/m64 ; 0F 0F /r 97
{[}3DNOW{]}}\end{flushleft}{\footnotesize \par}

\begin{flushleft}PFRSQRT calculates an approximation (accurate to
15 bits) of the reciprocal of the square root of the 32-bit floating-point
number in the lower half of the source and stores it in both halves
of the destination. If the source value is negative then it is treated
as positive except that its sign is copied to the result. \end{flushleft}

\begin{flushleft}\textbf{\large A.175 PFSUB: Packed Floating-Point
Subtraction}\end{flushleft}{\large \par}

\begin{flushleft}{\footnotesize PFSUB mmxreg,r/m64 ; 0F 0F /r 9A {[}3DNOW{]}}\end{flushleft}{\footnotesize \par}

\begin{flushleft}PFSUB subtracts the pair of 32-bit floating-point
numbers in the source operand from the corresponding pair in the destination
register. \end{flushleft}

\begin{flushleft}\textbf{\large A.176 PFSUBR: Packed Floating-Point
Reverse Subtraction}\end{flushleft}{\large \par}

\begin{flushleft}{\footnotesize PFSUBR mmxreg,r/m64 ; 0F 0F /r AA
{[}3DNOW{]}}\end{flushleft}{\footnotesize \par}

\begin{flushleft}PFSUB subtracts the pair of 32-bit floating-point
numbers in the source operand from the corresponding pair in the destination
register, and then negates the results. \end{flushleft}

\begin{flushleft}\textbf{\large A.177 PI2FD: Packed Integer To Floating-Point
Conversion}\end{flushleft}{\large \par}

\begin{flushleft}{\footnotesize PI2FD mmxreg,r/m64 ; 0F 0F /r 0D {[}3DNOW{]}}\end{flushleft}{\footnotesize \par}

\begin{flushleft}PI2FD converts the pair of signed 32-bit integers
in the source operand into 32-bit floating-point numbers (using truncation)
and stores the results in the destination register. \end{flushleft}

\begin{flushleft}\textbf{\large A.178 PI2FW: Packed Integer To Floating-Point
Conversion}\end{flushleft}{\large \par}

\begin{flushleft}{\footnotesize PI2FW mmxreg,r/m64 ; 0F 0F /r 0C {[}ATHLON{]}}\end{flushleft}{\footnotesize \par}

\begin{flushleft}PI2FW treats the source operand as a pair of signed
16-bit integers, by ignoring the upper 16 bits of each 32-bit half.
These integers are converted to 32-bit floating-point numbers and
stored in the destination register. \end{flushleft}

\begin{flushleft}\textbf{\large A.179 PINSRW: Insert Word}\end{flushleft}{\large \par}

\begin{flushleft}{\footnotesize PINSRW mmxreg,r/m16,imm8 ; 0F C4 /r
ib {[}KATMAI{]}}\end{flushleft}{\footnotesize \par}

\begin{flushleft}PINSRW loads a word from the 16-bit integer register
or memory location and inserts it in the MMX register, at a position
defined by the two least significant bits of the imm8 constant. \end{flushleft}

\begin{flushleft}Instead of specifying a 16-bit register you can instead
specify a 32-bit register, of which only the bottom half will be used.
This may seem peculiar, but it's the way Intel prefers. \end{flushleft}

\begin{flushleft}\textbf{\large A.180 PMACHRIW: MMX Packed Multiply
and Accumulate with Rounding}\end{flushleft}{\large \par}

\begin{flushleft}{\footnotesize PMACHRIW mmxreg,mem64 ; 0F 5E /r {[}CYRIX,MMX{]}}\end{flushleft}{\footnotesize \par}

\begin{flushleft}PMACHRIW acts almost identically to PMULHRIW (\uline{\textcolor[rgb]{0.000,0.000,1.000}{section A.189)}},
but instead of storing its result in the implied destination register,
it adds its result, as four packed words, to the implied destination
register. No saturation is done: the addition can wrap around. \end{flushleft}

\begin{flushleft}Note that PMACHRIW cannot take a register as its
second source operand. \end{flushleft}

\begin{flushleft}\textbf{\large A.181 PMADDWD: MMX Packed Multiply
and Add}\end{flushleft}{\large \par}

\begin{flushleft}{\footnotesize PMADDWD mmxreg,r/m64 ; 0F F5 /r {[}PENT,MMX{]}}\end{flushleft}{\footnotesize \par}

\begin{flushleft}PMADDWD treats its two inputs as vectors of four
signed words. It multiplies corresponding elements of the two operands,
giving four signed doubleword results. The top two of these are added
and placed in the top 32 bits of the destination (first) operand;
the bottom two are added and placed in the bottom 32 bits. \end{flushleft}

\begin{flushleft}\textbf{\large A.182 PMAGW: MMX Packed Magnitude}\end{flushleft}{\large \par}

\begin{flushleft}{\footnotesize PMAGW mmxreg,r/m64 ; 0F 52 /r {[}CYRIX,MMX{]}}\end{flushleft}{\footnotesize \par}

\begin{flushleft}PMAGW, specific to the Cyrix MMX extensions, treats
both its operands as vectors of four signed words. It compares the
absolute values of the words in corresponding positions, and sets
each word of the destination (first) operand to whichever of the two
words in that position had the larger absolute value. \end{flushleft}

\begin{flushleft}\textbf{\large A.183 PMAXSW: Packed Signed Integer
Word Maximum}\end{flushleft}{\large \par}

\begin{flushleft}{\footnotesize PMAXSW mmxreg,r/m64 ; 0F EE /r {[}KATMAI{]}}\end{flushleft}{\footnotesize \par}

\begin{flushleft}PMAXSW considers both operands as vectors of 4 signed
words. Each such word in the destination register is replaced by the
corresponding word in the source operand if that is larger. \end{flushleft}

\begin{flushleft}\textbf{\large A.184 PMAXUB: Packed Unsigned Integer
Byte Maximum}\end{flushleft}{\large \par}

\begin{flushleft}{\footnotesize PMAXUB mmxreg,r/m64 ; 0F DE /r {[}KATMAI{]}}\end{flushleft}{\footnotesize \par}

\begin{flushleft}PMAXUB considers both operands as vectors of 8 unsigned
bytes. Each such byte in the destination register is replaced by the
corresponding byte in the source operand if that is larger. \end{flushleft}

\begin{flushleft}\textbf{\large A.185 PMINSW: Packed Signed Integer
Word Minimum}\end{flushleft}{\large \par}

\begin{flushleft}{\footnotesize PMINSW mmxreg,r/m64 ; 0F EA /r {[}KATMAI{]}}\end{flushleft}{\footnotesize \par}

\begin{flushleft}PMINSW considers both operands as vectors of 4 signed
words. Each such word in the destination register is replaced by the
corresponding word in the source operand if that is smaller. \end{flushleft}

\begin{flushleft}\textbf{\large A.186 PMINUB: Packed Unsigned Integer
Byte Minimum}\end{flushleft}{\large \par}

\begin{flushleft}{\footnotesize PMINUB mmxreg,r/m64 ; 0F DA /r {[}KATMAI{]}}\end{flushleft}{\footnotesize \par}

\begin{flushleft}PMINUB considers both operands as vectors of 8 unsigned
bytes. Each such byte in the destination register is replaced by the
corresponding byte in the source operand if that is smaller. \end{flushleft}

\begin{flushleft}\textbf{\large A.187 PMOVMSKB: Move Byte Mask To
Integer}\end{flushleft}{\large \par}

\begin{flushleft}{\footnotesize PMOVMSKB reg32,mmxreg ; 0F D7 /r {[}KATMAI{]}}\end{flushleft}{\footnotesize \par}

\begin{flushleft}PMOVMSKB creates an 8-bit mask formed of the most
significant bit of each byte of its source operand, and stores this
in the destination register. \end{flushleft}

\begin{flushleft}\textbf{\large A.188 PMULHRWA: Packed Multiply With
Rounding}\end{flushleft}{\large \par}

\begin{flushleft}{\footnotesize PMULHRWA mmxreg,r/m64 ; 0F 0F /r B7
{[}3DNOW{]}}\end{flushleft}{\footnotesize \par}

\begin{flushleft}PMULHRWA performs the same operation as PMULHW, except
that it rounds the results rather than truncating. \end{flushleft}

\begin{flushleft}Note that the PMULHRWA instruction is called PMULHRW
in AMD documentation. But NASM uses the form PMULHRWA to avoid conflict
with Cyrix's different PMULHRW instruction (which NASM calls PMULHRWC). \end{flushleft}

\begin{flushleft}\textbf{\large A.189 PMULHRWC, PMULHRIW: MMX Packed
Multiply High with Rounding}\end{flushleft}{\large \par}

\begin{flushleft}{\footnotesize PMULHRWC mmxreg,r/m64 ; 0F 59 /r {[}CYRIX,MMX{]} }\end{flushleft}{\footnotesize \par}

\begin{flushleft}{\footnotesize PMULHRIW mmxreg,r/m64 ; 0F 5D /r {[}CYRIX,MMX{]}}\end{flushleft}{\footnotesize \par}

\begin{flushleft}These instructions, specific to the Cyrix MMX extensions,
treat their operands as vectors of four signed words. Words in corresponding
positions are multiplied, to give a 32-bit value in which bits 30
and 31 are guaranteed equal. Bits 30 to 15 of this value (bit mask
0x7FFF8000) are taken and stored in the corresponding position of
the destination operand, after first rounding the low bit (equivalent
to adding 0x4000 before extracting bits 30 to 15). \end{flushleft}

\begin{flushleft}For PMULHRWC, the destination operand is the first
operand; for PMULHRIW the destination operand is implied by the first
operand in the manner of PADDSIW (\uline{\textcolor[rgb]{0.000,0.000,1.000}{section A.152)}}. \end{flushleft}

\begin{flushleft}Note that the PMULHRWC instruction is called PMULHRW
in Cyrix documentation. But NASM uses the form PMULHRWC to avoid conflict
with AMD's different PMULHRW instruction (which NASM calls PMULHRWA) \end{flushleft}

\begin{flushleft}\textbf{\large A.190 PMULHUW: Packed Multiply High
Unsigned}\end{flushleft}{\large \par}

\begin{flushleft}{\footnotesize PMULHUW mmxreg,r/m64 ; 0F E4 /r {[}KATMAI{]}}\end{flushleft}{\footnotesize \par}

\begin{flushleft}PMULHUW multiplies the four unsigned words in the
destination register with the four unsigned words in the source operand.
The high-order 16 bits of each of the 32-bit intermediate results
are written to the destination operand. \end{flushleft}

\begin{flushleft}\textbf{\large A.191 PMULHW, PMULLW: MMX Packed Multiply}\end{flushleft}{\large \par}

\begin{flushleft}{\footnotesize PMULHW mmxreg,r/m64 ; 0F E5 /r {[}PENT,MMX{]} }\end{flushleft}{\footnotesize \par}

\begin{flushleft}{\footnotesize PMULLW mmxreg,r/m64 ; 0F D5 /r {[}PENT,MMX{]}}\end{flushleft}{\footnotesize \par}

\begin{flushleft}PMULxW treats its two inputs as vectors of four signed
words. It multiplies corresponding elements of the two operands, giving
four signed doubleword results. \end{flushleft}

\begin{flushleft}PMULHW then stores the top 16 bits of each doubleword
in the destination (first) operand; PMULLW stores the bottom 16 bits
of each doubleword in the destination operand. \end{flushleft}

\begin{flushleft}\textbf{\large A.192 PMVccZB: MMX Packed Conditional
Move}\end{flushleft}{\large \par}

\begin{flushleft}{\footnotesize PMVZB mmxreg,mem64 ; 0F 58 /r {[}CYRIX,MMX{]} }\end{flushleft}{\footnotesize \par}

\begin{flushleft}{\footnotesize PMVNZB mmxreg,mem64 ; 0F 5A /r {[}CYRIX,MMX{]} }\end{flushleft}{\footnotesize \par}

\begin{flushleft}{\footnotesize PMVLZB mmxreg,mem64 ; 0F 5B /r {[}CYRIX,MMX{]} }\end{flushleft}{\footnotesize \par}

\begin{flushleft}{\footnotesize PMVGEZB mmxreg,mem64 ; 0F 5C /r {[}CYRIX,MMX{]}}\end{flushleft}{\footnotesize \par}

\begin{flushleft}These instructions, specific to the Cyrix MMX extensions,
perform parallel conditional moves. The two input operands are treated
as vectors of eight bytes. Each byte of the destination (first) operand
is either written from the corresponding byte of the source (second)
operand, or left alone, depending on the value of the byte in the
implied operand (specified in the same way as PADDSIW, in \uline{\textcolor[rgb]{0.000,0.000,1.000}{section A.152)}}. \end{flushleft}

\begin{flushleft}PMVZB performs each move if the corresponding byte
in the implied operand is zero. PMVNZB moves if the byte is non-zero.
PMVLZB moves if the byte is less than zero, and PMVGEZB moves if the
byte is greater than or equal to zero. \end{flushleft}

\begin{flushleft}Note that these instructions cannot take a register
as their second source operand. \end{flushleft}

\begin{flushleft}\textbf{\large A.196 POR: MMX Bitwise OR}\end{flushleft}{\large \par}

\begin{flushleft}{\footnotesize POR mmxreg,r/m64 ; 0F EB /r {[}PENT,MMX{]}}\end{flushleft}{\footnotesize \par}

\begin{flushleft}POR performs a bitwise OR operation between its two
operands (i.e. each bit of the result is 1 if and only if at least
one of the corresponding bits of the two inputs was 1), and stores
the result in the destination (first) operand. \end{flushleft}

\begin{flushleft}\textbf{\large A.197 PREFETCH, PREFETCHW: Prefetch
cache line}\end{flushleft}{\large \par}

\begin{flushleft}{\footnotesize PREFETCH mem ; 0F 0D /0 {[}3DNOW{]} }\end{flushleft}{\footnotesize \par}

\begin{flushleft}{\footnotesize PREFETCHW mem ; 0F 0D /1 {[}3DNOW{]}}\end{flushleft}{\footnotesize \par}

\begin{flushleft}PREFETCH loads a cache line into the L1 data cache.
PREFETCHW does the same, but also marks the cache line as modified. \end{flushleft}

\begin{flushleft}\textbf{\large A.198 PREFETCHNTA, PREFETCHT0, PREFETCHT1,
PREFETCHT2: Prefetch cache line}\end{flushleft}{\large \par}

\begin{flushleft}{\footnotesize PREFETCHNTA mem ; 0F 18 /0 {[}KATMAI{]} }\end{flushleft}{\footnotesize \par}

\begin{flushleft}{\footnotesize PREFETCHT0 mem ; 0F 18 /1 {[}KATMAI{]} }\end{flushleft}{\footnotesize \par}

\begin{flushleft}{\footnotesize PREFETCHT1 mem ; 0F 18 /2 {[}KATMAI{]} }\end{flushleft}{\footnotesize \par}

\begin{flushleft}{\footnotesize PREFETCHT2 mem ; 0F 18 /3 {[}KATMAI{]}}\end{flushleft}{\footnotesize \par}

\begin{flushleft}These instructions move the data specified by the
address closer to the processor using, respectively, the nta, t0,
t1 and t2 hints. \end{flushleft}

\begin{flushleft}\textbf{\large A.199 PSADBW: Packed Sum of Absolute
Differences}\end{flushleft}{\large \par}

\begin{flushleft}{\footnotesize PSADBW mmxreg,r/m64 ; 0F F6 /r {[}KATMAI{]}}\end{flushleft}{\footnotesize \par}

\begin{flushleft}PSADBW computes the sum of the absolute differences
of the unsigned signed bytes in the destination register and those
in the source operand. It then places this sum in the lowest word
of the destination register, and sets the three other words to zero. \end{flushleft}

\begin{flushleft}\textbf{\large A.200 PSHUFW: Packed Shuffle Word}\end{flushleft}{\large \par}

\begin{flushleft}{\footnotesize PSHUFW mmxreg,r/m64,imm8 ; 0F 70 /r
ib {[}KATMAI{]}}\end{flushleft}{\footnotesize \par}

\begin{flushleft}PSHUFW uses the imm8 value to select which of the
four words of the source operand will be placed in each of the words
of the destination register. Bits 0 and 1 of imm8 encode the source
for word 0 (the lowest word) of the destination register, bits 2 and
3 encode the source for word 1, bits 4 and 5 for word 2, and bits
6 and 7 for word 3 (the highest word). Each 2-bit encoding is a number
in the range 0-3 that specifies the corresponding word of the source
operand. \end{flushleft}

\begin{flushleft}\textbf{\large A.201 PSLLx, PSRLx, PSRAx: MMX Bit
Shifts}\end{flushleft}{\large \par}

\begin{flushleft}{\footnotesize PSLLW mmxreg,r/m64 ; 0F F1 /r {[}PENT,MMX{]} }\end{flushleft}{\footnotesize \par}

\begin{flushleft}{\footnotesize PSLLW mmxreg,imm8 ; 0F 71 /6 ib {[}PENT,MMX{]}}\end{flushleft}{\footnotesize \par}

\begin{flushleft}{\footnotesize PSLLD mmxreg,r/m64 ; 0F F2 /r {[}PENT,MMX{]} }\end{flushleft}{\footnotesize \par}

\begin{flushleft}{\footnotesize PSLLD mmxreg,imm8 ; 0F 72 /6 ib {[}PENT,MMX{]}}\end{flushleft}{\footnotesize \par}

\begin{flushleft}{\footnotesize PSLLQ mmxreg,r/m64 ; 0F F3 /r {[}PENT,MMX{]} }\end{flushleft}{\footnotesize \par}

\begin{flushleft}{\footnotesize PSLLQ mmxreg,imm8 ; 0F 73 /6 ib {[}PENT,MMX{]}}\end{flushleft}{\footnotesize \par}

\begin{flushleft}{\footnotesize PSRAW mmxreg,r/m64 ; 0F E1 /r {[}PENT,MMX{]} }\end{flushleft}{\footnotesize \par}

\begin{flushleft}{\footnotesize PSRAW mmxreg,imm8 ; 0F 71 /4 ib {[}PENT,MMX{]}}\end{flushleft}{\footnotesize \par}

\begin{flushleft}{\footnotesize PSRAD mmxreg,r/m64 ; 0F E2 /r {[}PENT,MMX{]} }\end{flushleft}{\footnotesize \par}

\begin{flushleft}{\footnotesize PSRAD mmxreg,imm8 ; 0F 72 /4 ib {[}PENT,MMX{]}}\end{flushleft}{\footnotesize \par}

\begin{flushleft}{\footnotesize PSRLW mmxreg,r/m64 ; 0F D1 /r {[}PENT,MMX{]} }\end{flushleft}{\footnotesize \par}

\begin{flushleft}{\footnotesize PSRLW mmxreg,imm8 ; 0F 71 /2 ib {[}PENT,MMX{]}}\end{flushleft}{\footnotesize \par}

\begin{flushleft}{\footnotesize PSRLD mmxreg,r/m64 ; 0F D2 /r {[}PENT,MMX{]} }\end{flushleft}{\footnotesize \par}

\begin{flushleft}{\footnotesize PSRLD mmxreg,imm8 ; 0F 72 /2 ib {[}PENT,MMX{]}}\end{flushleft}{\footnotesize \par}

\begin{flushleft}{\footnotesize PSRLQ mmxreg,r/m64 ; 0F D3 /r {[}PENT,MMX{]} }\end{flushleft}{\footnotesize \par}

\begin{flushleft}{\footnotesize PSRLQ mmxreg,imm8 ; 0F 73 /2 ib {[}PENT,MMX{]}}\end{flushleft}{\footnotesize \par}

\begin{flushleft}PSxxQ perform simple bit shifts on the 64-bit MMX
registers: the destination (first) operand is shifted left or right
by the number of bits given in the source (second) operand, and the
vacated bits are filled in with zeros (for a logical shift) or copies
of the original sign bit (for an arithmetic right shift). \end{flushleft}

\begin{flushleft}PSxxW and PSxxD perform packed bit shifts: the destination
operand is treated as a vector of four words or two doublewords, and
each element is shifted individually, so bits shifted out of one element
do not interfere with empty bits coming into the next. \end{flushleft}

\begin{flushleft}PSLLx and PSRLx perform logical shifts: the vacated
bits at one end of the shifted number are filled with zeros. PSRAx
performs an arithmetic right shift: the vacated bits at the top of
the shifted number are filled with copies of the original top (sign)
bit. \end{flushleft}

\begin{flushleft}\textbf{\large A.202 PSUBxx: MMX Packed Subtraction}\end{flushleft}{\large \par}

\begin{flushleft}{\footnotesize PSUBB mmxreg,r/m64 ; 0F F8 /r {[}PENT,MMX{]} }\end{flushleft}{\footnotesize \par}

\begin{flushleft}{\footnotesize PSUBW mmxreg,r/m64 ; 0F F9 /r {[}PENT,MMX{]} }\end{flushleft}{\footnotesize \par}

\begin{flushleft}{\footnotesize PSUBD mmxreg,r/m64 ; 0F FA /r {[}PENT,MMX{]}}\end{flushleft}{\footnotesize \par}

\begin{flushleft}{\footnotesize PSUBSB mmxreg,r/m64 ; 0F E8 /r {[}PENT,MMX{]} }\end{flushleft}{\footnotesize \par}

\begin{flushleft}{\footnotesize PSUBSW mmxreg,r/m64 ; 0F E9 /r {[}PENT,MMX{]}}\end{flushleft}{\footnotesize \par}

\begin{flushleft}{\footnotesize PSUBUSB mmxreg,r/m64 ; 0F D8 /r {[}PENT,MMX{]} }\end{flushleft}{\footnotesize \par}

\begin{flushleft}{\footnotesize PSUBUSW mmxreg,r/m64 ; 0F D9 /r {[}PENT,MMX{]}}\end{flushleft}{\footnotesize \par}

\begin{flushleft}PSUBxx all perform packed subtraction between their
two 64-bit operands, storing the result in the destination (first)
operand. The PSUBxB forms treat the 64-bit operands as vectors of
eight bytes, and subtract each byte individually; PSUBxW treat the
operands as vectors of four words; and PSUBD treats its operands as
vectors of two doublewords. \end{flushleft}

\begin{flushleft}In all cases, the elements of the operand on the
right are subtracted from the corresponding elements of the operand
on the left, not the other way round. \end{flushleft}

\begin{flushleft}PSUBSB and PSUBSW perform signed saturation on the
sum of each pair of bytes or words: if the result of a subtraction
is too large or too small to fit into a signed byte or word result,
it is clipped (saturated) to the largest or smallest value which will
fit. PSUBUSB and PSUBUSW similarly perform unsigned saturation, clipping
to 0FFh or 0FFFFh if the result is larger than that. \end{flushleft}

\begin{flushleft}\textbf{\large A.203 PSUBSIW: MMX Packed Subtract
with Saturation to Implied Destination}\end{flushleft}{\large \par}

\begin{flushleft}{\footnotesize PSUBSIW mmxreg,r/m64 ; 0F 55 /r {[}CYRIX,MMX{]}}\end{flushleft}{\footnotesize \par}

\begin{flushleft}PSUBSIW, specific to the Cyrix extensions to the
MMX instruction set, performs the same function as PSUBSW, except
that the result is not placed in the register specified by the first
operand, but instead in the implied destination register, specified
as for PADDSIW (\uline{\textcolor[rgb]{0.000,0.000,1.000}{section A.152)}}. \end{flushleft}

\begin{flushleft}\textbf{\large A.204 PSWAPD: Packed Swap Doubleword}\end{flushleft}{\large \par}

\begin{flushleft}{\footnotesize PSWAPD mmxreg,r/m64 ; 0F 0F /r BB
{[}ATHLON{]}}\end{flushleft}{\footnotesize \par}

\begin{flushleft}PSWAPD copies the source operand to the destination
register, swapping the upper and lower halves in the process. \end{flushleft}

\begin{flushleft}\textbf{\large A.205 PUNPCKxxx: Unpack Data}\end{flushleft}{\large \par}

\begin{flushleft}{\footnotesize PUNPCKHBW mmxreg,r/m64 ; 0F 68 /r
{[}PENT,MMX{]} }\end{flushleft}{\footnotesize \par}

\begin{flushleft}{\footnotesize PUNPCKHWD mmxreg,r/m64 ; 0F 69 /r
{[}PENT,MMX{]} }\end{flushleft}{\footnotesize \par}

\begin{flushleft}{\footnotesize PUNPCKHDQ mmxreg,r/m64 ; 0F 6A /r
{[}PENT,MMX{]}}\end{flushleft}{\footnotesize \par}

\begin{flushleft}{\footnotesize PUNPCKLBW mmxreg,r/m64 ; 0F 60 /r
{[}PENT,MMX{]} }\end{flushleft}{\footnotesize \par}

\begin{flushleft}{\footnotesize PUNPCKLWD mmxreg,r/m64 ; 0F 61 /r
{[}PENT,MMX{]} }\end{flushleft}{\footnotesize \par}

\begin{flushleft}{\footnotesize PUNPCKLDQ mmxreg,r/m64 ; 0F 62 /r
{[}PENT,MMX{]}}\end{flushleft}{\footnotesize \par}

\begin{flushleft}PUNPCKxx all treat their operands as vectors, and
produce a new vector generated by interleaving elements from the two
inputs. The PUNPCKHxx instructions start by throwing away the bottom
half of each input operand, and the PUNPCKLxx instructions throw away
the top half. \end{flushleft}

\begin{flushleft}The remaining elements, totalling 64 bits, are then
interleaved into the destination, alternating elements from the second
(source) operand and the first (destination) operand: so the leftmost
element in the result always comes from the second operand, and the
rightmost from the destination. \end{flushleft}

\begin{flushleft}PUNPCKxBW works a byte at a time, PUNPCKxWD a word
at a time, and PUNPCKxDQ a doubleword at a time. \end{flushleft}

\begin{flushleft}So, for example, if the first operand held 0x7A6A5A4A3A2A1A0A
and the second held 0x7B6B5B4B3B2B1B0B, then: \end{flushleft}

\begin{flushleft}PUNPCKHBW would return 0x7B7A6B6A5B5A4B4A. \end{flushleft}

\begin{flushleft}PUNPCKHWD would return 0x7B6B7A6A5B4B5A4A. \end{flushleft}

\begin{flushleft}PUNPCKHDQ would return 0x7B6B5B4B7A6A5A4A. \end{flushleft}

\begin{flushleft}PUNPCKLBW would return 0x3B3A2B2A1B1A0B0A. \end{flushleft}

\begin{flushleft}PUNPCKLWD would return 0x3B2B3A2A1B0B1A0A. \end{flushleft}

\begin{flushleft}PUNPCKLDQ would return 0x3B2B1B0B3A2A1A0A. \end{flushleft}

\begin{flushleft}\textbf{\large A.209 PXOR: MMX Bitwise XOR}\end{flushleft}{\large \par}

\begin{flushleft}{\footnotesize PXOR mmxreg,r/m64 ; 0F EF /r {[}PENT,MMX{]}}\end{flushleft}{\footnotesize \par}

\begin{flushleft}PXOR performs a bitwise XOR operation between its
two operands (i.e. each bit of the result is 1 if and only if exactly
one of the corresponding bits of the two inputs was 1), and stores
the result in the destination (first) operand. \end{flushleft}

\begin{flushleft}\textbf{\large A.211 RCPPS: SSE Packed Single-FP
Reciprocal Approximation}\end{flushleft}{\large \par}

\begin{flushleft}{\footnotesize RCPPS xmmreg,r/m128 ; 0F 53 /r {[}KATMAI,SSE{]}}\end{flushleft}{\footnotesize \par}

\begin{flushleft}Four each of the four 32-bit floating-point numbers
in the source operand RCPPS calculates an approximation of the reciprocal
and stores it in the corresponding quarter of the destination register.
The absolute value of the error for each of these approximations is
at most 3/8192. \end{flushleft}

\begin{flushleft}\textbf{\large A.212 RCPSS: SSE Scalar Single-FP
Reciprocal Approximation}\end{flushleft}{\large \par}

\begin{flushleft}{\footnotesize RCPSS xmmreg,xmmreg/mem32 ; F3 0F
53 /r {[}KATMAI,SSE{]}}\end{flushleft}{\footnotesize \par}

\begin{flushleft}RCPSS calculates an approximation of the reciprocal
of the 32-bit floating-point in the source operand (using the lowest
quarter of the source operand if it is a register) and places the
result in the lowest quarter of the destination register. The absolute
value of the error for this approximation is at most 3/8192. \end{flushleft}

\begin{flushleft}\textbf{\large A.222 RSQRTPS: Packed Single-FP Square
Root Reciprocal}\end{flushleft}{\large \par}

\begin{flushleft}{\footnotesize RSQRTPS xmmreg,r/m128 ; 0F 52 /r {[}KATMAI,SSE{]}}\end{flushleft}{\footnotesize \par}

\begin{flushleft}For each of the four 32-bit floating-point numbers
in the source operand, RSQRTPS computes an approximation of the reciprocal
of the square root, and puts this in the corresponding quarter of
the destination register. The maximum absolute error for this approximation
is 3/8192. \end{flushleft}

\begin{flushleft}\textbf{\large A.223 RSQRTSS:Scalar Single-FP Square
Root Reciprocal}\end{flushleft}{\large \par}

\begin{flushleft}{\footnotesize RSQRTSS xmmreg,r/m128 ; F3 0F 52 /r
{[}KATMAI,SSE{]}}\end{flushleft}{\footnotesize \par}

\begin{flushleft}RSQRTSS computes an approximation of the reciprocal
of the square root of the first 32-bit floating-point number from
xmm2/m32 and puts it in the lowest quarter of the destination register.
The maximum absolute error for this approximation is 3/8192. \end{flushleft}

\begin{flushleft}\textbf{\large A.231 SFENCE: Store Fence}\end{flushleft}{\large \par}

\begin{flushleft}{\footnotesize SFENCE ; 0F AE /7 {[}KATMAI{]}}\end{flushleft}{\footnotesize \par}

\begin{flushleft}SFENCE guarantees that all store instructions which
precede it in the program order are globally visible before any store
instructions which follow it. \end{flushleft}

\begin{flushleft}\textbf{\large A.235 SHUFPS: Shuffle Single-FP}\end{flushleft}{\large \par}

\begin{flushleft}{\footnotesize SHUFPS xmmreg,r/m128,imm8 ; 0F C6
/r ib {[}KATMAI,SSE{]}}\end{flushleft}{\footnotesize \par}

\begin{flushleft}SHUFPS copies two quarters of the destination register
to the lower two quarters of the destination register, and copies
two quarters of the source operand to the upper two quarters of the
destination register. \end{flushleft}

\begin{flushleft}Bits 0 and 1 of imm8 determine which of the four
quarters of the destination register gets copied to the lowest quater
of the destination register. Bits 2 and 3 of imm8 similary deterine
which quarter is copied to the second quarter of the destination register.
Bits 4-7 likewise select the quarters of the source operand to be
copied. \end{flushleft}

\begin{flushleft}\textbf{\large A.239 SQRTPS: Packed Single-FP Square
Root}\end{flushleft}{\large \par}

\begin{flushleft}{\footnotesize SQRTPS xmmreg,r/m128 ; 0F 51 /r {[}KATMAI,SSE{]}}\end{flushleft}{\footnotesize \par}

\begin{flushleft}SQRTPS considers the source operand as a vector of
four 32-bit floating-point numbers, and for each of these it computes
the square root and stores the result in the corresponding quarter
of the destination register. \end{flushleft}

\begin{flushleft}\textbf{\large A.240 SQRTSS: Scalar Single-FP Square
Root}\end{flushleft}{\large \par}

\begin{flushleft}{\footnotesize SQRTSS xmmreg,xmmreg/mem32 ; F3 0F
51 /r {[}KATMAI,SSE{]}}\end{flushleft}{\footnotesize \par}

\begin{flushleft}SQRTSS computes the square root of the 32-bit floating-point
number in the lowest quarter of the source operand and stores the
result in the lowest quarter of the destination register. \end{flushleft}

\begin{flushleft}\textbf{\large A.242 STMXCSR: SSE Store MXCSR}\end{flushleft}{\large \par}

\begin{flushleft}{\footnotesize STMXCSR mem32 ; 0F AE /3 {[}KATMAI,SSE{]}}\end{flushleft}{\footnotesize \par}

\begin{flushleft}STMXCSR copies the the MXCSR (the SSE control/status
register) into the 32-bit memory location. \end{flushleft}

\begin{flushleft}\textbf{\large A.246 SUBPS: Packed Single-FP Subtract}\end{flushleft}{\large \par}

\begin{flushleft}{\footnotesize SUBPS xmmreg,r/m128 ; 0F 5C /r {[}KATMAI,SSE{]}}\end{flushleft}{\footnotesize \par}

\begin{flushleft}SUBPS considers both operands as vectors of four
32-bit floating-point numbers, and subtracts each such number in the
source operand from the corresponding number in the destination register. \end{flushleft}

\begin{flushleft}\textbf{\large A.247 SUBSS: Scalar Single-FP Subtract}\end{flushleft}{\large \par}

\begin{flushleft}{\footnotesize SUBSS xmmreg,xmmreg/mem32 ; F3 0F
5C /r {[}KATMAI,SSE{]}}\end{flushleft}{\footnotesize \par}

\begin{flushleft}SUBSS subtracts the 32-bit floating-point number
in the lowest 4 bytes of the source operand from the corresponding
number in the destination register. \end{flushleft}

\begin{flushleft}\textbf{\large A.256 UCOMISS: Unordered Scalar Single-FP
Compare and set EFLAGS}\end{flushleft}{\large \par}

\begin{flushleft}{\footnotesize UCOMISS xmmreg,xmmreg/mem32 ; 0F 2E
/r {[}KATMAI,SSE{]}}\end{flushleft}{\footnotesize \par}

\begin{flushleft}UCOMISS compares the 32-bit floating-point numbers
in the lowest part of the two operands, and sets the CPU flags appropriately.
UCOMISS differs from COMISS in that it signals an invalid numeric
exeception only if an operand is an sNaN, whereas COMISS does so also
if an operand is a qNaN. \end{flushleft}

\begin{flushleft}\textbf{\large A.259 UNPCKHPS: Unpack High Packed
Single-FP Data}\end{flushleft}{\large \par}

\begin{flushleft}{\footnotesize UNPCKHPS xmmreg,r/m128 ; 0F 15 /r
{[}KATMAI,SSE{]}}\end{flushleft}{\footnotesize \par}

\begin{flushleft}UNPCKHPS performs an interleaved unpack of the high-order
data elements of the two operands in the following manner: labelling
the data elements of the destination register as X0, X1, X2 and X3
(from low to high) and those of the source operand as Y0, Y1, Y2 and
Y3 the UNPCKHPS instruction simultaneously performs the four assignments
X0 := X2, X1 := Y2, X2 := X3 and X3 := Y3. \end{flushleft}

\begin{flushleft}\textbf{\large A.260 UNPCKLPS: Unpack Low Packed
Single-FP Data}\end{flushleft}{\large \par}

\begin{flushleft}{\footnotesize UNPCKLPS xmmreg,r/m128 ; 0F 14 /r
{[}KATMAI,SSE{]}}\end{flushleft}{\footnotesize \par}

\begin{flushleft}UNPCKLPS performs an interleaved unpack of the low-order
data elements of the two operands in the following manner: labelling
the data elements of the destination register as X0, X1, X2 and X3
(from low to high) and those of the source operand as Y0, Y1, Y2 and
Y3 the UNPCKLPS instruction simultaneously performs the four assignments
X0 := X0, X1 := Y0, X2 := X1 and X3 := Y1. \end{flushleft}

\begin{flushleft}\textbf{\large A.269 XLATB: Translate Byte in Lookup
Table}\end{flushleft}{\large \par}

\begin{flushleft}{\footnotesize XLATB ; D7 {[}8086{]}}\end{flushleft}{\footnotesize \par}

\begin{flushleft}XLATB adds the value in AL, treated as an unsigned
byte, to BX or EBX, and loads the byte from the resulting address
(in the segment specified by DS) back into AL. \end{flushleft}

\begin{flushleft}The base register used is BX if the address size
is 16 bits, and EBX if it is 32 bits. If you need to use an address
size not equal to the current BITS setting, you can use an explicit
a16 or a32 prefix. \end{flushleft}

\begin{flushleft}The segment register used to load from {[}BX+AL{]}
or {[}EBX+AL{]} can be overridden by using a segment register name
as a prefix (for example, es xlatb). \end{flushleft}

\begin{flushleft}\textbf{\large A.271 XORPS: SSE Bitwise Logical XOR}\end{flushleft}{\large \par}

\begin{flushleft}{\footnotesize XORPS xmmreg,r/m128 ; 0F 57 /r {[}KATMAI,SSE{]}}\end{flushleft}{\footnotesize \par}

\begin{flushleft}XORPS performs a bitwise XOR operation on the source
operand and the destination register, and stores the result in the
destination register. \end{flushleft}

\begin{center}\uline{\textcolor[rgb]{0.000,0.000,1.000}{Previous Chapter }}|
\uline{\textcolor[rgb]{0.000,0.000,1.000}{Contents }}| \uline{\textcolor[rgb]{0.000,0.000,1.000}{Index }}\end{center}
\end{document}
